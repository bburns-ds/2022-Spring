\documentclass{article}
\usepackage{bbpsets}

\lhead{Ben Burns}
\chead{\textbf{Math 621}}
\rhead{Due: February 6th, 2022}
\pagestyle{fancy}

\begin{document}
\parskip=0em
\begin{mdframed}[backgroundcolor=blue!20]
\textbf{Problem 1}: Let $a, b \in \C$ and $|a| < r < |b|$ Let $\gamma$ be a circle of radius $r$ centered at the origin. Evaluate
\begin{center}
    $\int_{\gamma} \dfrac{dz}{(z-a)(z-b)}$
\end{center}
(Use only the definition of the integral but not Cauchy theorem or residues)
\end{mdframed}


\parskip=0em
\begin{mdframed}[backgroundcolor=blue!20]
\textbf{Problem 2}: Let $\gamma_R^+$ be an upper semicircle of radius $R$ centeret at the origin. Show that 
\begin{center}
    $\int_{\gamma_R^+}\dfrac{1-e^{iz}}{z^2}dz \underset{R\rightarrow 0}{\rightarrow} 0  $
\end{center}
\end{mdframed}


\parskip=0em
\begin{mdframed}[backgroundcolor=blue!20]
\textbf{Problem 3}: Recall that an open set $\Omega \subset \C$ is called connected if it cannot be expressed as a union of disjoing non-empty open sets. Show that $\Omega$ is connected if and only if every two points $z_1, z_2 \in \Omega$ can be connected by a polygonal path $\gamma$, i.e. a piece-wise smooth curve that consists of finitely many straight line segments.
\end{mdframed}


\parskip=0em
\begin{mdframed}[backgroundcolor=blue!20]
\textbf{Problem 4}: Suppose $f$ is holomorphic in $\Omega \in \C$ and $Re(f)$ is constant. Prove that $f$ is locally constant. Is it necessarily constant?
\end{mdframed}


\parskip=0em
\begin{mdframed}[backgroundcolor=blue!20]
\textbf{Problem 5}: Let $\D$ be a the (open) unit disc and fix $w \in \D$. Consider the function $F(z) = \dfrac{w-z}{1-\bar{w}z}$. Prove that $F$ is a bjiective holomorphic function $\D \to \D$. 
\end{mdframed}


\parskip=0em
\begin{mdframed}[backgroundcolor=blue!20]
\textbf{Problem 6}: (a) Show that the Cauchy-Riemann equations take the following form in polar coordinates: 
\begin{center}
    $\dfrac{\partial u}{\partial r} = \dfrac{1}{r}\dfrac{\partial v}{\partial \theta}$ and $\dfrac{\partial v}{\partial r} = -\dfrac{1}{r}\dfrac{\partial u}{\partial \theta}$
\end{center}
(b) Use (a) to show that the logarithm function defined as $\log(z) = \log(r) + i\theta$ is holomorphic for $r > 0, -\pi < \theta < \pi$
\end{mdframed}


\parskip=0em
\begin{mdframed}[backgroundcolor=blue!20]
\textbf{Problem 7}: Let $\Delta = \dfrac{\partial^2}{\partial x^2} + \dfrac{\partial^2}{\partial y^2}$ be the Laplacian. Show that $\Delta = 4\dfrac{\partial}{\partial z}\dfrac{\partial}{\partial \bar{z}} = 4 \dfrac{\partial}{\partial \bar{z}}\dfrac{\partial}{\partial z}$
\end{mdframed}


\parskip=0em
\begin{mdframed}[backgroundcolor=blue!20]
\textbf{Problem 8}: (a) Let $\alpha_n$ be a sequence of positive real numbers such that $\lim\limits_{n\to \infty} \dfrac{\alpha_{n+1}}{\alpha_n} = L.$\\ Prove: $\lim\limits_{n \to \infty} a_n^{1/n} = L$\\
SS: In particular, this exercise shows that when applicable, the ratio test can be used to calculate the radius of convergence of a power series.\\
(b) Use (a) to compute radius of convergence of hypergeometric series 
\begin{center}
    $1 + \sum\limits_{n = 1}^\infty \dfrac{\alpha(\alpha + 1)\cdots(a + n-1)\beta(\beta + 1)\cdots(\beta + n-1)}{n!\gamma(\gamma + 1)\cdots(\gamma + n - 1)}z^n$
\end{center}
Here $\alpha, \beta, \gamma \in \C$ and $\gamma \neq 0, -1, -2, \ldots$
\end{mdframed}


\parskip=0em
\begin{mdframed}[backgroundcolor=blue!20]
\textbf{Problem 9}: Prove that\\
(a) $\sum\limits_{n \geq 0} nz^n$ does not converge at any point of the unit circle\\
(b) $\sum\limits_{n\geq 1}\dfrac{z^n}{n^2}$ converges at every point of the unit circle
\end{mdframed}


\parskip=0em
\begin{mdframed}[backgroundcolor=blue!20]
\textbf{Problem 10}: Let $f$ be a power seies centered at the origin. Prove that $f$ has a power series expansion around any point in its disc of convergence.
\end{mdframed}


\end{document}