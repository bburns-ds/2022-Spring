\documentclass{article}
\usepackage{bbpsets}

\lhead{Ben Burns}
\chead{\textbf{Math 621}}
\rhead{Due: February 6th, 2022}
\pagestyle{fancy}

\begin{document}
% Todo: 4, 6a

\newpage\parskip=0em
\begin{mdframed}[backgroundcolor=blue!20]
\textbf{Problem 1}: Let $a, b \in \C$ and $|a| < r < |b|$ Let $\gamma$ be a circle of radius $r$ centered at the origin. Evaluate
\begin{center}
    $\int_{\gamma} \dfrac{dz}{(z-a)(z-b)}$
\end{center}
(Use only the definition of the integral but not Cauchy theorem or residues)
\end{mdframed}
\textbf{Solution}\parskip=1em\\
First, note that because $|a| < r$ and $|z| = r$, $|z| > 0$. Using partial fraction decomposition of the integrand,
\begin{align*}
    \dfrac{1}{(z-a)(z-b)} &= \dfrac{A}{z-a} + \dfrac{B}{z-b}\\
    Az -Ab &+ Bz -ab = 1\\
    A + B = 0 &, A = -B\\
    B(b-a) &= 1\\
    \dfrac{1}{(z-a)(z-b)} &=  \dfrac{1}{(a-b)(z-a)} - \dfrac{1}{(a-b)(z-b)}\\
\end{align*}
Taking the integral 
\begin{align*}
    \int_{\gamma} \dfrac{dz}{(z-a)(z-b)} &= \int_\gamma \dfrac{1}{(a-b)(z-a)} - \dfrac{1}{(a-b)(z-b)}dz\\
    &= \dfrac{1}{a-b}\left[\int_\gamma \dfrac{1}{z-a}dz - \int_\gamma\dfrac{1}{z-b}dz\right]\\
    &= \dfrac{1}{a-b}\left[\int_\gamma \dfrac{1}{z}\dfrac{1}{1-a/z}dz   - \int_\gamma-\dfrac{1}{b}\dfrac{1}{1-z/b}dz\right]\\
    \sum\limits_{n \geq 0}ar^{n} &= \dfrac{a}{1-r}, |r| < 1\\
    |a| < r \implies |a| &< |z| \implies \dfrac{|a|}{|z|} < 1\\
    r < |b| \implies |z| &< |b| \implies \dfrac{|z|}{|b|} < 1\\
    \int_{\gamma} \dfrac{dz}{(z-a)(z-b)} &= \dfrac{1}{a-b}\left[\int_\gamma \dfrac{1}{z}\sum\limits_{n\geq 0}(\dfrac{a}{z})^ndz   + \int_\gamma\dfrac{1}{b}\sum\limits_{n\geq 0}(\dfrac{z}{b})^ndz\right]\\
    &= \dfrac{1}{a-b}\left[\int_\gamma \sum\limits_{n\geq 0}\dfrac{1}{z}(\dfrac{a}{z})^ndz   + \int_\gamma\sum\limits_{n\geq 0} \dfrac{1}{b}(\dfrac{z}{b})^ndz\right]\\
\end{align*}
From the third to last and second to last lines, the series both converge absolutely, so the order of summation and integration can be interchanged.
\begin{align*}
    \int_{\gamma} \dfrac{dz}{(z-a)(z-b)}&= \dfrac{1}{a-b}\left[ \sum\limits_{n\geq 0}\int_\gamma\dfrac{1}{z}(\dfrac{a}{z})^ndz   + \sum\limits_{n\geq 0} \int_\gamma \dfrac{1}{b}(\dfrac{z}{b})^ndz\right]\\
    &= \dfrac{1}{a-b}\left[ \sum\limits_{n\geq 0}a^n\int_\gamma\dfrac{1}{z^{n+1}}dz   + \sum\limits_{n\geq 0} \dfrac{1}{b^{n+1}}\int_\gamma z^ndz\right]\\
\end{align*}
$f(z) = z^n$ has primitive $F(z) = \dfrac{z^{n+1}}{n+1}$, where $n \neq -1$. By Corollary 3.3 in Stein Shakarchi:

\textbf{Corollary 3.3} \emph{If $\gamma$ is a closed curve in an open set $\Omega$, and $f$ is continuous and has a primitive in $\Omega$, then}
\begin{center}
    $\int_\gamma f(z) dz = 0$
\end{center}
The second integral will always be continuous and have a primitive, and therefore evaluates to 0 for all $n \geq 0$, so the second sum evaluates to 0. Additionally, because $|z| > 0$, $f(z)$ is continuous in its defined set $\Omega$, and the first integral is also continuous with a primitive where $n > 0$ ($n$ in the context of the first summation, which results in the exponent of $z$ being -1). Therefore, all terms of the first summation for $n > 0$ are 0, leaving just 
\begin{align*}
    \int_{\gamma} \dfrac{dz}{(z-a)(z-b)} &= \dfrac{1}{a-b}\int_\gamma\dfrac{1}{z}dz
\end{align*}
Using the parameterization $z = re^{i\theta}$, we can evaluate this exactly as 
\begin{align*}
    \int_{\gamma} \dfrac{dz}{(z-a)(z-b)} &= \dfrac{1}{a-b}\int_{0}^{2\pi}\dfrac{1}{re^{i\theta}}ire^{i\theta}d\theta\\
&= \dfrac{i}{a-b}\int_0^{2\pi} d\theta\\
&= \dfrac{2\pi i}{a-b}\\
\end{align*}
\newpage\parskip=0em
\begin{mdframed}[backgroundcolor=blue!20]
\textbf{Problem 2}: Let $\gamma_R^+$ be an upper semicircle of radius $R$ centeret at the origin. Show that 
\begin{center}
    $\int_{\gamma_R^+}\dfrac{1-e^{iz}}{z^2}dz \underset{R\rightarrow 0}{\rightarrow} 0  $
\end{center}
\end{mdframed}
\textbf{Solution}\parskip=1em\\
Using the regular $z = Re^{i\theta}$ polar substitution:
\begin{align*}
    \int_{\gamma_R^+}\dfrac{1-e^{iz}}{z^2}dz &= \int_{0}^{\pi}\dfrac{1-e^{iRe^{i\theta}}}{(Re^{i\theta})^2}iRe^{i\theta}dz\\
    &= \int_{0}^{\pi}\dfrac{1-e^{iRe^{i\theta}}}{Re^{i\theta}}idz\\
    &= \dfrac{i}{R}\int_{0}^{\pi}\dfrac{1-e^{iRe^{i\theta}}}{e^{i\theta}}dz\\
\end{align*}
Now, all three terms in the integrand are bounded over a finite arc length, so we can conclude that the integral is bounded. As $R \to \infty$, $(\dfrac{i}{R} \cdot \text{bounded}) \to 0$. Therefore $\int_{\gamma_R^+}\dfrac{1-e^{iz}}{z^2}dz \to 0$ as $R \to \infty$

\newpage\parskip=0em
\begin{mdframed}[backgroundcolor=blue!20]
\textbf{Problem 3}: Recall that an open set $\Omega \subset \C$ is called connected if it cannot be expressed as a union of disjoing non-empty open sets. Show that $\Omega$ is connected if and only if every two points $z_1, z_2 \in \Omega$ can be connected by a polygonal path $\gamma$, i.e. a piece-wise smooth curve that consists of finitely many straight line segments.
\end{mdframed}
\textbf{Solution}\parskip=1em\\
I am assuming that the pathwise connecting curve $\gamma$ has to be entirely contained in $\Omega$ as it is defined in Stein Shakarchi, else this biconditional is not true. Otherwise, you could just take any two points in two distant, disjoint, non-empty subsets and join them with a curve that goes through $\C$. Two points in a connected set $\Omega \subset \C$ can be pathwise connected, as is proved below, so the statement to be proved would only be a left to right implication.

$\impliedby$:\\
Assuming both that $\Omega \subset \C$ is open and that every two points $z_1, z_2 \in \Omega$ can be connected by a polygonal path $\gamma$, i.e. a piece-wise smooth curve that consists of finitely many straight line segments, we claim that there exist disjoint, non-empty open sets $\Omega_1, \Omega_2 \subset \C$ such that $\Omega_1 \cup \Omega_2 = \Omega$, and will arrive at a contradiction.

Fix two arbitrary points $w_1 \in \Omega_1$ and $w_2 \in \Omega_2$.  Define $\gamma:[0, 1] \to \Omega$ to be such a polygonal path that connects $w_1$ and $w_2$ such that $\gamma(0) = w_1$ and $\gamma(1) = w_2$. Since $\gamma$ is smooth, each point is uniquely defined by a value in [0, 1], and since [0, 1] is continuous, every point along the path has a value that maps to it, meaning $\gamma$ has an inverse. Define the intervals $\gamma^{-1}(\Omega_1)$ and $\gamma^{-1}(\Omega_2)$, the subintervals of $[0, 1]$ that $\gamma$ maps to points in $\Omega_1$ and $\Omega_2$ respectively. These sets must be disjoint, since $\Omega_1 \cap \Omega_2 = \emptyset$ (and no single value in the parameterization can define two separate points). Additionally, neither interval is non-empty: $\gamma^{-1}(\Omega_1)$ contains $\gamma^{-1}(w_1)$ and $\gamma^{-1}(w_2)$. Finally, $\gamma^{-1}(\Omega_1) \cup \gamma^{-1}(\Omega_2) = [0, 1]$, because $\gamma \subset \Omega$. 

Therefore the interval [0, 1] is \emph{not} connected, which is a contradiction (since every interval in $\R$ is connected), and our assumption that $\Omega$ is not connected is incorrect.

$\implies$: \\
Assume that $\Omega$ is an open, connected, non-empty set in $\C$. Fix a point $w \in \Omega$. We claim that there exists some point $v\in \Omega$ such that there is no path in $\Omega$ connecting $w$ and $v$, and want to arrive at a contradiction.

Define $\Omega_1 \subset \Omega$ as the set of all points that can be connected to $w$ by a polygonal path in $\Omega$, and $\Omega_2 \subset \Omega$ to be all points that cannot be connected to $w$. Our goal is to show that $\Omega_2$ must be empty $\implies \not\exists v \in \Omega$.

$\Omega_1$ and $\Omega_2$ are open: assuming $\Omega_1$ is closed, take an arbitrary boundary point $x_1$ (which can be connected to $w$ by finitely many line segments) with a neighborhood including a point $x_2$ in $\Omega$ but not in $\Omega_1$. There has to exist such $x_1$, $x_2$, else $\Omega$ contains all its boundary points. We could then connect $x_1$ to $x_2$ with a straight line segment $x_2 - x_1 \in \C$. Now $x_2$ can be connected to $w$ with finitely many line segments, meaning $x_2$ must be in $\Omega_1$, and $w_1$ is no longer a boundary point. This can be done for all boundary points of $\Omega_1$ (so long as the diameter of the neighborhood does not exit $\Omega$), therefore it is open. The reverse logic implies $\Omega_2$ is open: if $\Omega_2$ has a boundary point that cannot be connected, there is a point in some neighorhood in $\Omega$ that includes a point that \emph{can} be connected to $w$, therefore we connect them by the finite line segment represented by the subtraction of the two points, and the boundary point is no longer in $\Omega_2$, again making it open. (this conceptual gets us to conclude that $\Omega_1 = \Omega$, but we can be more through with the rest of the proof)

If $\Omega_1$ and $\Omega_2$ are obviously disjoint, since a possible path from $w$ to a fixed candidate point can't both exist and not exist. All points in both $\Omega_1$ and $\Omega_2$ are in $\Omega$ by definition, so $\Omega_1 \cup \Omega_2 \subset \Omega$. Likewise, all points in $\Omega$ must be in either $\Omega_1$ or $\Omega_2$, since a fixed point can't (again) not have a path and have a path to $w$, $\implies \Omega = \Omega_1 \cup \Omega_2$. 

$w \in \Omega_1 \implies \Omega_1$ nonempty (the trivial path connects $w$ to itself). $\Omega_2$ also being non-empty contradicted $\Omega$ being connected, therefore $\Omega_2$ must be empty $\implies \Omega_1 = \Omega \implies \Omega$ is pathwise connected.
\newpage\parskip=0em
\begin{mdframed}[backgroundcolor=blue!20]
\textbf{Problem 4}: Suppose $f$ is holomorphic in $\Omega \in \C$ and $Re(f)$ is constant. Prove that $f$ is locally constant. Is it necessarily constant?
\end{mdframed}
\textbf{Solution}\parskip=1em\\
Define $z = x + iy$ and $f = u(x, y) + iv(u, y)$. 

$Re(f) \implies u(x, y)$ constant $\implies \dfrac{\partial u}{\partial x} = \dfrac{\partial u}{\partial y} = 0$.\\ 
$f$ is holomorphic $\implies$ $f$ satisfies the Cauchy Riemann equations $\implies \dfrac{\partial u}{\partial x} = \dfrac{\partial v}{\partial y}$ and $\dfrac{\partial u}{\partial y} = -\dfrac{\partial v}{\partial x} \implies 0 = \dfrac{\partial v}{\partial y}$ and $0 = -\dfrac{\partial v}{\partial x} \implies \dfrac{\partial v}{\partial x} = \dfrac{\partial v}{\partial y}.$

Because all four partials are constant, $f' = 0$, and is locally constant. Therefore, $f$ is constant in each connected component/region, and by Corollary 3.4 in Stein Shakarchi, this is sufficient to call $f$ constant in each connected component. 

However, $f$ isn't necessarily constant overall. In their proof of Corollary 3.4, Stein Shakarchi uses $f(w) = f(w_0)$ for fixed $w$ in the region and $w_0$ arbitrary in the region as their necessary condition for "constant". Because $\Omega$ isn't ever constained to be connected, we can conceive a disconnected $\Omega = \Omega_1 \cup \Omega_2 \cup...\cup \Omega_n$ for all non-empty open sets and all pairwise disjoint. $f$ can be constant in each  region (Corollary 3.4 definition), with the same global real value, but with each region can take different complex values, making $f$ not constant (proof condition). 

This construction never contradicts the partial derivatives of $v(x, y)$ being 0 in any isolated connected component. Such a contradiction would require us to construct a path between the two points in two different regions to show "global" change in $v(x, y)$, and likely take an integral along with the path with the two points as bounds, and obtain a non-zero result, as in the proof of 3.4. However, we now know that such a path is impossible to construct given $\Omega$ is not connected, using the condition proven in quesiton 3 of this homework.

\newpage\parskip=0em
\begin{mdframed}[backgroundcolor=blue!20]
\textbf{Problem 5}: Let $\D$ be a the (open) unit disc and fix $w \in \D$. Consider the function $F(z) = \dfrac{w-z}{1-\bar{w}z}$. Prove that $F$ is a bjiective holomorphic function $\D \to \D$. 
\end{mdframed}
\textbf{Solution}\parskip=1em\\
To show $F(z)$ is a bijective function $\D \to \D$, it suffices to show that $F(z)$ is its own inverse, is defined in all of $\D$, and actually maps elements of $\D$ to $\D$. First, showing bijection:
\begin{align*}
    F(F(z)) &= \dfrac{w-\dfrac{w-z}{1-\bar{w}z}}{1-\bar{w}\dfrac{w-z}{1-\bar{w}z}}\\
    &= \dfrac{\dfrac{w-w\bar{w}z -w+z}{1-\bar{w}z}}{\dfrac{1-\bar{w}z-\bar{w}w+\bar{w}z}{1-\bar{w}z}}\\
    &= \dfrac{w-w\bar{w}z -w+z}{1-\bar{w}z-\bar{w}w+\bar{w}z}\\
    &= \dfrac{w\bar{w}z+z}{1-\bar{w}w}\\
    &= \dfrac{z(w\bar{w}+1)}{1-\bar{w}w}\\
    &= z
\end{align*}
The division by $1-\bar{w}z$ on the third line is valid because, for $\bar{w}z = 1$ for variable $w, z\in \D$ $w$ fixed, $|\bar{w}| = |w|< 1$, so $|\bar{w}||z| = 1 \implies |z| = \dfrac{1}{|w|} > 1$, therefore $z \not\in \D$, so $1 -\bar{w}{z} \neq 0$ for $w, z \in \D$. 

Assuming $w, z \in \D$ and $F(z) \not \in \D \implies |F(z)| = \dfrac{|w-z|}{|1-\bar{w}z|} \geq 1$. Applying triangle inequality

\begin{align*}
    \dfrac{\left|w+(-z)\right|}{\left|1+(-\bar{w}z)\right|} &\leq \dfrac{|w|+|z|}{1+|\bar{w}z|}\\
    1 &\leq \dfrac{|w|+|z|}{1+|w||z|}\\
    1+|w||z| &\leq |w|+|z|\\
    1 - |w| - |z| + |w||z| &\leq 0\\
    (1 - |w|)(1 - |z|) &\leq 0\\
\end{align*}
Since $w$ is fixed in $\D$, we know that $|w| < 1 \implies 0 < 1 - |w|$. Therefore $1-|z| \leq 0 \implies |z| \geq 1$ independant of which $z$ and $w$ we choose in $\D$. Therefore, our assumption that $|F(z)| > 1$ is incorrect, and $|F(z)| < 1 \implies F(z) \in \D$ for $w, z \in \D$.

$F(z)$ is holomophic because both the numerator, $f_1$, and deonminator, $f_2$, are holomorphic in $\D$.
\begin{align*}
    \lim\limits_{h\to 0} \dfrac{w - (z + h) - (w + z)}{h} &= \lim\limits_{h\to 0} \dfrac{w - z - h - w + z}{h}\\
    &= \lim\limits_{h\to 0} \dfrac{h}{h}\\
    &= 1
\end{align*}
\begin{align*}
    \lim\limits_{h\to 0} \dfrac{1-\bar{w}(z+h)- (1-\bar{w}z)}{h} &= \lim\limits_{h\to 0} \dfrac{1-\bar{w}z+\bar{w}h- 1+\bar{w}z}{h}\\
    &= \lim\limits_{h\to 0} \dfrac{\bar{w}h}{h}\\
    &= \bar{w}\\
\end{align*}
$F = f_1/f_2$ is holomorphic in $\D$ so long as $f_2(z_0) \neq 0$ for any $z_0 \in \D$. As was said after the bijective proof, $1-\bar{w}z \neq 0$ unless $z \not \in \D$, therefore $\forall z_0 \in \D: f_2(z_0) \neq 0$, and $F$ is a bijective holomorphic function $\D \to \D$.
\newpage\parskip=0em
\begin{mdframed}[backgroundcolor=blue!20]
\textbf{Problem 6}: (a) Show that the Cauchy-Riemann equations take the following form in polar coordinates: 
\begin{center}
    $\dfrac{\partial u}{\partial r} = \dfrac{1}{r}\dfrac{\partial v}{\partial \theta}$ and $\dfrac{\partial v}{\partial r} = -\dfrac{1}{r}\dfrac{\partial u}{\partial \theta}$
\end{center}
(b) Use (a) to show that the logarithm function defined as $\log(z) = \log(r) + i\theta$ is holomorphic for $r > 0, -\pi < \theta < \pi$
\end{mdframed}
\textbf{Solution}\parskip=1em\\
(a) I don't know if you wanted us to plug in/switch variables to polar at an intermediary step using previous Cauchy-Riemann identities (and I missed office hours/didn't ask), so rather than risk that I chose to do the whole thing from the top.

Define $z = re^{i\theta}$ and $f(r, \theta) = u(r, \theta) + iv(r, \theta)$. Fix $z_0 \in C$ with $r > $ and move $r$ towards $r_0$. Moving $r$ along the angle of $r_0$, 
\begin{align*}
    f'(z_0) = \lim\limits_{r\to 0} \dfrac{f(re^{i\theta_0}) - f(r_0e^{i\theta_0})}{r}\\
    \lim\limits_{r\to 0} \dfrac{u(r, \theta_0) + iv(r, \theta_0) - u(r_0, \theta_0) - iv(r_0, \theta_0)}{r}\\
    \lim\limits_{r\to 0} \dfrac{u(r, \theta_0)  - u(r_0, \theta_0) }{r}+ i\dfrac{v(r, \theta_0)- v(r_0, \theta_0)}{r}\\
    \left[\lim\limits_{r\to 0} \dfrac{u(r, \theta_0)  - u(r_0, \theta_0) }{r}+ i\dfrac{v(r, \theta_0)- v(r_0, \theta_0)}{r}\right]\\
    \left[\left(\lim\limits_{r\to 0} \dfrac{u(r, \theta_0)  - u(r_0, \theta_0) }{r}\right)+ i\left(\lim\limits_{r\to 0}\dfrac{v(r, \theta_0)- v(r_0, \theta_0)}{r}\right)\right]\\
    \left[\dfrac{\partial u}{\partial r} + i \dfrac{\partial v}{\partial r}\right]\\
\end{align*}

Now fixing the radius $r_0$, fixing $\theta_0$, and moving along the circle as $\theta \to 0$ gives

\begin{align*}
    f'(z_0) = \lim\limits_{\theta\to 0} \dfrac{f(r_0e^{i\theta}) - f(r_0e^{i\theta_0})}{\theta}\\
    r_0^{-1}\lim\limits_{\theta\to 0} \dfrac{u(r_0, \theta) + iv(r_0, \theta) - u(r_0, \theta_0) - iv(r_0, \theta_0)}{\theta}\\
    r_0^{-1}\lim\limits_{\theta\to 0}\left[ \dfrac{u(r_0, \theta) + iv(r_0, \theta) - u(r_0, \theta_0) - iv(r_0, \theta_0)}{\theta}\right]\\
    r_0^{-1}\left[\lim\limits_{\theta\to 0}\left(\dfrac{u(r_0, \theta) - u(r_0, \theta_0)}{\theta}\right) + i\left(\lim\limits_{\theta\to 0}\dfrac{v(r_0, \theta)  - v(r_0, \theta_0)}{\theta}\right)\right]\\
    \left[\dfrac{1}{r}\dfrac{\partial v}{\partial \theta}-\dfrac{i}{r}\dfrac{\partial u}{\partial \theta}\right]\\
\end{align*}

As in the proof for our standard basis Cauchy-Riemann equations, we set the real and imaginary parts of our two coexisting defitions of $f'(z_0)$ to be equal, and obtain $\dfrac{\partial u}{\partial r} = \dfrac{1}{r}\dfrac{\partial v}{\partial \theta}$ and $\dfrac{\partial v}{\partial r} = \dfrac{1}{r}\dfrac{\partial u}{\partial \theta}$

(b) $\log(z) = \log(re^{i\theta}) = \log(z) + \log(e^{i\theta}) = \log(z) + i\theta$. 
$r_0 <0$ is equivalent to $r_1 = -r_0 > 0, \theta_1 = -\theta_0$. 

Using the polar Cauchy-Riemann equations, $\dfrac{\partial u}{\partial r} = \dfrac{\partial }{\partial r}[\log(r)] = \dfrac{1}{r}$, and $\dfrac{\partial v}{\partial \theta} = \dfrac{\partial }{\partial \theta}[\theta] = 1$. Therefore $\dfrac{1}{r} = \dfrac{1}{r}\cdot 1 \implies \dfrac{\partial u}{\partial r} = \dfrac{1}{r}\dfrac{\partial v}{\partial \theta}$. 

Note that we further constrain $r \neq 0$, since $\dfrac{\partial u}{\partial r} = \dfrac{1}{r}$ does not exist where $r = 0$.

Further, both $\dfrac{\partial v}{\partial r} = \dfrac{\partial }{\partial r}[\theta] = 0$ and $\dfrac{\partial u}{\partial \theta} = \dfrac{\partial }{\partial \theta}[\log(r)] = 0$, satisfying $\dfrac{\partial v}{\partial r} = -\dfrac{1}{r}\dfrac{\partial u}{\partial \theta}$. 

Therefore, we can use the biconditional that $f$ is holomorphic at $z_0 \iff f$ satisfies the Cauchy-Riemann equations at $z_0$ to say that $f$ is holomorphic for $r > 0$ and $-\pi < \theta < \pi$. 

Note the additional constraint on $\theta$: we must constrain $\theta$ to not include $\pi$ and $-\pi$ (the negative real axis). If we fix $r$, approaching the negative real axis counterclockwise: $\lim\limits_{\theta \to \pi} \dfrac{\partial v}{\partial \theta} = \pi$, but approaching clockwise gives $\lim\limits_{\theta\to -\pi}\dfrac{\partial v}{\partial \theta} = -\pi$, therefore $f$ cannot be holomorphic along the negative real axis. This also prevents us from continuing to rotate to contradict holomorphic at other points.
\newpage\parskip=0em
\begin{mdframed}[backgroundcolor=blue!20]
\textbf{Problem 7}: Let $\Delta = \dfrac{\partial^2}{\partial x^2} $ be the Laplacian. Show that $\Delta = 4\dfrac{\partial}{\partial z}\dfrac{\partial}{\partial \bar{z}} = 4 \dfrac{\partial}{\partial \bar{z}}\dfrac{\partial}{\partial z}$
\end{mdframed}
\textbf{Solution}\parskip=1em\\
\begin{align*}
    \Delta &= \dfrac{\partial^2}{\partial x^2}+ \dfrac{\partial^2}{\partial y^2}\\
    &= \dfrac{\partial^2}{\partial x^2}- \dfrac{1}{i^2}\dfrac{\partial^2}{\partial y^2}\\
    &= (\dfrac{\partial}{\partial x}+ \dfrac{1}{i}\dfrac{\partial}{\partial y})(\dfrac{\partial}{\partial x}- \dfrac{1}{i}\dfrac{\partial}{\partial y})\\
    &= 4\left[\dfrac{1}{2}(\dfrac{\partial}{\partial x}+ \dfrac{1}{i}\dfrac{\partial}{\partial y})\dfrac{1}{2}(\dfrac{\partial}{\partial x}- \dfrac{1}{i}\dfrac{\partial}{\partial y})\right]\\
    &= 4\dfrac{\partial}{\partial z}\dfrac{\partial}{\partial \bar{z}}\\
    &= 4\dfrac{\partial}{\partial \bar{z}}\dfrac{\partial}{\partial z}
\end{align*}
The last line assumes continuous second partial derivatives, but we can do this because otherwise the claim of equality in the problem does not hold.
\newpage\parskip=0em
\begin{mdframed}[backgroundcolor=blue!20]
\textbf{Problem 8}: (a) Let $\alpha_n$ be a sequence of positive real numbers such that $\lim\limits_{n\to \infty} \dfrac{\alpha_{n+1}}{\alpha_n} = L.$\\ Prove: $\lim\limits_{n \to \infty} a_n^{1/n} = L$\\
SS: In particular, this exercise shows that when applicable, the ratio test can be used to calculate the radius of convergence of a power series.\\
(b) Use (a) to compute radius of convergence of hypergeometric series 
\begin{center}
    $1 + \sum\limits_{n = 1}^\infty \dfrac{\alpha(\alpha + 1)\cdots(a + n-1)\beta(\beta + 1)\cdots(\beta + n-1)}{n!\gamma(\gamma + 1)\cdots(\gamma + n - 1)}z^n$
\end{center}
Here $\alpha, \beta, \gamma \in \C$ and $\gamma \neq 0, -1, -2, \ldots$
\end{mdframed}
\textbf{Solution}\parskip=1em\\
(a) Because $\alpha_n$ is a sequence of positive reals, we can express the same limit with absolute values
\begin{align*}
    \lim\limits_{n\to \infty} \dfrac{\alpha_{n+1}}{\alpha_n} &= \lim\limits_{n\to \infty} \dfrac{|\alpha_{n+1}|}{|\alpha_n|} = L\\
    \forall \epsilon > 0: \exists N : \forall n \geq N &: \left|\dfrac{|\alpha_{n+1}|}{|\alpha_n|} - L \right| < \epsilon
\end{align*}
Now note that we can reexpress the domininator $|\alpha_n|$ as 
\begin{align*}
    |a_n| &= |a_n|\left|\dfrac{a_{n-1}}{a_{n-1}}\right| \cdots \left|\dfrac{a_N}{a_N}\right|\\
    &= \left|\dfrac{a_{n}}{a_{n-1}}\right|\left|\dfrac{a_{n-1}}{a_{n-2}}\right| \cdots \left|\dfrac{a_N+1}{a_N}\right|\left|a_N\right|\\
\end{align*}
Each fractional term in this second product is a value less than $L+\epsilon$, so 
\begin{align*}
    \left|\dfrac{\alpha_{n+1}}{\alpha_n}\right| < L + \epsilon\\
\end{align*}
Therefore
\begin{align*}
    |a_n| &= \left|\dfrac{a_{n}}{a_{n-1}}\right|\left|\dfrac{a_{n-1}}{a_{n-2}}\right| \cdots \left|\dfrac{a_N+1}{a_N}\right|\left|a_N\right|< (L+\epsilon)^{n-N}\left|a_N\right|\\
    \implies |a_n|^{1/n} &< (L + \epsilon)^{1 - \dfrac{N}{n}}|a_N|^{1/n}\\
\end{align*}
Taking the limit $n\to \infty$, $\dfrac{N}{n} \to 0$ and $|a_N|^{1/n} \to 1$, and $|a_n|^{1/n} = a_n^{a/n}$ since $a_n$ positive,

\begin{align*}
    a_n^{1/n} &< L + \epsilon\\
    \implies a_n^{1/n} - L &< \epsilon\\
    \implies \left|a_n^{1/n} - L\right| &< \epsilon\\
    \therefore \lim\limits_{n\to \infty} a_n^{1/n} = L\\
\end{align*}

(b) The general form of $a_n$ is 
\begin{align*}
    a_n = \dfrac{\alpha(\alpha + 1)\cdots(\alpha + n-1)\beta(\beta + 1)\cdots(\beta + n-1)}{n!\gamma(\gamma + 1)\cdots(\gamma + n - 1)}
\end{align*}
meaning the ratio $a_{n+1}/a_n$ takes 
\begin{align*}
    \dfrac{a_{n+1}}{a_n} &= \dfrac{\dfrac{\alpha(\alpha + 1)\cdots(\alpha + n-1)(\alpha+n)\beta(\beta + 1)\cdots(\beta + n-1)(\beta+n)}{(n+1)!\gamma(\gamma + 1)\cdots(\gamma + n - 1)(\gamma+n)}}{\dfrac{\alpha(\alpha + 1)\cdots(\alpha + n-1)\beta(\beta + 1)\cdots(\beta + n-1)}{n!\gamma(\gamma + 1)\cdots(\gamma + n - 1)}}\\
    &= \dfrac{(\alpha+n)(\beta + n)}{(n+1)(\gamma + n)}\\
    &= \dfrac{(\dfrac{\alpha}{n}+1)(\dfrac{\beta}{n} + 1)}{(1+\dfrac{1}{n})(\dfrac{\gamma}{n} + 1)}\\
    \dfrac{|a_{n+1}|}{|a_n|}&= \dfrac{\left|(\dfrac{\alpha}{n}+1)(\dfrac{\beta}{n} + 1)\right|}{\left|(1+\dfrac{1}{n})(\dfrac{\gamma}{n} + 1)\right|}\\
\end{align*}
Taking the limit, 
\begin{align*}
    \lim\limits_{n\to \infty}\dfrac{\left|(\dfrac{\alpha}{n}+1)(\dfrac{\beta}{n} + 1)\right|}{\left|(1+\dfrac{1}{n})(\dfrac{\gamma}{n} + 1)\right|}&= 1\\
    \lim\limits_{n\to \infty}\dfrac{\left|\dfrac{\alpha}{n}+1\right|\left|\dfrac{\beta}{n} + 1\right|}{\left|1+\dfrac{1}{n}\right|\left|\dfrac{\gamma}{n} + 1\right|}&= 1\\
\end{align*}
Because taking the absolute value gives a sequence of positive reals (since norm is a real value and products of positive reals are positive reals), we can use part (a) to say that 
\begin{align*}
    \lim\limits_{n\to \infty} |a_n|^{1/n} &= 1\\
    \implies \limsup\limits_{n\to \infty} |a_n|^{1/n} &= 1\\
    \implies \dfrac{1}{R} &= 1\\
    \therefore R &= 1\\
\end{align*}
\newpage\parskip=0em
\begin{mdframed}[backgroundcolor=blue!20]
\textbf{Problem 9}: Prove that\\
(a) $\sum\limits_{n \geq 0} nz^n$ does not converge at any point of the unit circle\\
(b) $\sum\limits_{n\geq 1}\dfrac{z^n}{n^2}$ converges at every point of the unit circle
\end{mdframed}
\textbf{Solution}\parskip=1em\\
(a) Recall that $\sum\limits_{n\geq 0}a_n$ converges $\iff$ $\lim\limits_{n \to \infty} a_n = 0$. $|z| = 1 \implies \lim\limits_{n\to\infty}|nz^n| = n$, which does not tend towards 0. Therefore $\sum\limits_{n\geq 0}nz^n$ diverges for all $z$ such that $|z| = 1$

(b) Define $\sum\limits{n\geq 1} \dfrac{1}{n^2}$. Recall the comparison test, that if $\sum b_n$ converges, and $0 \leq a_n \leq b_n$ for sufficiently large $n$, then $\sum a_n$ also converges. Because $\zeta(2) = \sum\limits_{n\geq 1} \dfrac{1}{n^2}$ converges to $\dfrac{\pi^2}{6}$ and $|z| = 1 \implies \forall n \geq 1: |nz^n| = 1$, which is less than or equal to $|b_n| = 1$ for all such $n$, therefore $\sum\limits_{n \geq 1}\dfrac{z^n}{n^2}$ converges.

\newpage\parskip=0em
\begin{mdframed}[backgroundcolor=blue!20]
\textbf{Problem 10}: Let $f$ be a power series centered at the origin. Prove that $f$ has a power series expansion around any point in its disc of convergence.
\end{mdframed}
\textbf{Solution}\parskip=1em\\
First, state 
\begin{align*}
    f(z) &= \sum\limits_{n\geq 0} a_nz^n
\end{align*}
Taking the hint from Stein Shakarchi, we rexpressed $z = z_0 + (z - z_0)$, where $z_0$ is an arbitrary point in the disc of convergence of $f$: $|z_0| < R$. Using this substitution in the definition of $f$, we can expand the power term using the binomial theorem:
\begin{align*}
    \sum\limits_{n\geq 0} a_nz^n &= \sum\limits_{n\geq 0} a_n(z_0 + (z - z_0))^n\\
    &= \sum\limits_{n\geq 0} a_n\sum\limits_{0 \leq k \leq n}{n\choose k} (z_0)^{n-k}(z - z_0)^k\\
    &= \sum\limits_{n\geq 0}\sum\limits_{0 \leq k \leq n} a_n{n\choose k} (z_0)^{n-k}(z - z_0)^k\\
\end{align*}
Because $\sum\limits_{n \geq 0}a_nz^n$ is absolutely convergent, we can commute terms and/or swap summations. Observe that if we swap the summations so that we first iterate the sum over $k$, and then iterate the inner sum over $n$, our values for $k$ will take values $k\geq 0$, and $n$ will only take values $n\geq k$
\begin{align*}
    \sum\limits_{n\geq 0}\sum\limits_{0 \leq k \leq n} a_n{n\choose k} (z_0)^{n-k}(z - z_0)^k &= \sum\limits_{k\geq 0}\sum\limits_{n \geq k} a_n{n\choose k} (z_0)^{n-k}(z - z_0)^k\\
    %&= \sum\limits_{k\geq 0}b_k(z - z_0)^k\\
    %b_k &= \sum\limits_{n \geq k} a_n{n\choose k} (z_0)^{n-k}\\
\end{align*}
% $b_0 = \sum\limits_{n \geq 0}a_n{n\choose 0}(z_0)^{n} = \sum\limits_{n \geq 0}a_nz_0^{n} = f(z_0)$ exists because $z_0$ was chosen such that $|z| < R$. For general $b_k$, 
% \begin{align*}
%     b_k &= \sum\limits_{n \geq k} a_n{n\choose k} (z_0)^{n-k}\\
%     &= \sum\limits_{n \geq 0} a_n{n\choose k} (z_0)^{n-k} - \sum\limits_{0 \leq n \leq k-1} a_n{n\choose k} (z_0)^{n-k}\\
%     &= \dfrac{1}{k!}\sum\limits_{n \geq 0} \dfrac{n!}{(n-k)!}a_n(z_0)^{n-k} - \sum\limits_{0 \leq n \leq k-1} a_n{n\choose k} (z_0)^{n-k}\\
%     &= \dfrac{1}{k!}f^{(k)}(z_0) - \sum\limits_{0 \leq n \leq k-1} a_n{n\choose k} (z_0)^{n-k}\\
% \end{align*}
% $f^{(k)}(z_0)$ also converges since $|z_0| < R$.
%Todo: just do absolute convergence
Since $|z_0| < R$ and $z$ must be in the disc of convergence as well, $|z - z_0| < R - |z_0|$ (geometrically, it is necessary for convergence that the component of $z$'s norm along the vector defined by $z_0$ cannot go farther out than $R-|z_0|$, else $|z| > R$.) In other words, $|z - z_0| + |z_0|< R$. Therefore, to show absolute convergence, 
\begin{align*}
    \sum\limits_{k\geq 0}\sum\limits_{n \geq k} \left|a_n{n\choose k} (z_0)^{n-k}(z - z_0)^k\right| &= \sum\limits_{k\geq 0}\sum\limits_{n \geq k} |a_n||{n\choose k}|| (z_0)^{n-k}||(z - z_0)^k|\\
    &= \sum\limits_{k\geq 0}\sum\limits_{n \geq k} |a_n|{n\choose k} (|z_0|)^{n-k}(|z - z_0|)^k\\
    &= \sum\limits_{k\geq 0}|a_n|\sum\limits_{n \geq k} {n\choose k} (|z_0|)^{n-k}(|z - z_0|)^k\\
    &= \sum\limits_{k\geq 0}|a_n|(|z_0|+|z - z_0|)^k\\
\end{align*}
Because $|z_0|+|z - z_0| < R$, this series converges absolutely, and therefore converges, so $\sum\limits_{k\geq 0}\sum\limits_{n \geq k} a_n{n\choose k} (z_0)^{n-k}(z - z_0)^k$ converges for $|z_0| < R$, and $f$ has a power series expansion around any point in its disc of convergence.
\end{document}