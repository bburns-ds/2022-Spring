\documentclass{article}
\usepackage{bbpsets}

\lhead{Ben Burns}
\chead{\textbf{Math 412}}
\rhead{Due: February 6th, 2022}
\pagestyle{fancy}

\begin{document}

\newpage\parskip=0em
\begin{mdframed}[backgroundcolor=blue!20]
\textbf{Problem 1}: Describe all homomorphisms from a given ring $R$ to a given ring $S$ explicitly, i.e. say where every element $r\in R$ goes to in $S$. Prove that your functions are indeed homomorphisms and that there are no other homomorphisms.
\begin{itemize}
    \item $R = \Z$, $S = \Z \times \Z$
    \item $R = \Z_5, S = \Q$
    \item $R = \Z_2 \times \Z_2, S = \Z_2$
\end{itemize}
\end{mdframed}
\textbf{Solution}\parskip=1em\\
\textbf{Preface}: A general strategy that I will be using in my solutions is using the fundamental theorem on homomorphisms: 

(a) Define $\phi: \Z \to \Z\times \Z$ as $\phi(z) = (z, 0)$. To prove homomorphism: \\
$\phi(r + s) = (r + s, 0) = (r, 0) + (s, 0) = \phi(r) + \phi(s)$\\
$\phi(rs) = (rs, 0) = (r, 0)(s, 0) = \phi(r)\phi(s)$.\\
$\phi(z) = (0, z)$ works in the same way.

Another homomorphism would be $\phi(z) = \phi(z, z)$. This is a homomorphism:
$\phi(r + s) = (r + s, r + s) = (r, r) + (s, s) = \phi(r) + \phi(s)$\\
$\phi(rs) = (rs, rs) = (r, r)(s, s) = \phi(r)\phi(s)$.

Lastly, $\phi(z) = (0, 0)$ works trivially just as $\psi: \Z \to \Z, \psi(z) = 0$ is a homomorphism.

There are no others, because the additive 0 in $R$ must map to $(0, 0)$ in $S$. Linearly setting a coordinate to anything other than $z$ or 0 will mean 0 does not transfer in this manner (needs to be linear so addition holds after applying the homomorphism)

(b) The trivial homomorphism of mapping all elements in $\Z_5$ to 0 in $\Q$ will work.

For nontrivial, $0, 1\in \Z_5$ must map to $0, 1 \in \Q$. For addition to hold, adding 1 to itself 4 times (4 additions so 5 ones) must result in getting 0 in $\Z_5$, which will then be mapped to 0 $\in \Q$. However, we must also be able to apply the homomorphism before adding, so we need to map 1 to an element in $\Q$ that, when added to itself 5 times, obtains 0. However, we have already mapped $1_{\Z_5}$ to $1_\Q$ (which is required for additive homomorphism to hold), so since $1_\Q$ does not behave in this way, there is no valid homomorphism, because we can not obtain the multiplicative identity from repeated addition of the additive identity in $\Q$ like we can in $\Z_5$. We basically can't make the characteristics of the two rings cooperate with each other.

(c) Define $\phi : \Z_2 \times \Z_2 \to \Z_2$ to be 
\begin{align*}
    (0, 0) &\to 0\\  
    (0, 1) &\to 1\\
    (1, 0) &\to 1\\
    (1, 1) &\to 1
\end{align*}
Which looks a lot like an xor gate. In other words, $\phi((a, b)) = a + b \in \Z_2$.

$\Z_2 \times \Z_2$ and $\Z_2$ are commutative rings (since both are abelian under both operations from 411), so both $+$ and $\cdot$ are associative and commutative in our ring (axiom)
\begin{align*}
    \phi((a, b) + (a', b')) &= \phi((a + a', b + b'))\\
    &= (a + a') + (b + b')\\
    &= a + (a' + b) + b'\\ 
    &= a + (b + a') + b'\\ 
    &= (a + b) + (a' + b')\\
    &= \phi((a, b)) + \phi((a', b'))
\end{align*}
\begin{align*}
    \phi((a, b)(a', b')) &= \phi((aa', bb'))\\
    &= (aa')(bb')\\
    &= a(a'b)b'\\
    &= a(ba')b'\\
    &= (ab)(a'b)'\\
    &= \phi((a,b))\phi((a',b))'\\
\end{align*}
There are no other homomorphisms
\newpage\parskip=0em
\begin{mdframed}[backgroundcolor=blue!20]
\textbf{Problem 2}: For a given subset $S$ of a given ring $R$, decide whether $S$ is a subring or not (with proof)
\begin{itemize}
    \item $S = \{ a + b\sqrt{2} | a, b \in\Z\}, R = \R$
    \item $S = \{ f(x) | f'(3) = 0\}, R = \{f : \R \to \R\}$
\end{itemize}
\end{mdframed}
\textbf{Solution}\parskip=1em\\
(a) $S$ is closed, because $(a + b\sqrt2) + (a' + b'\sqrt2) = (a + a') + (b+b')\sqrt2$, and $(a + b\sqrt2)(a' + b'\sqrt2) = (aa' + 2bb') + (a'b + ab')\sqrt2$. The elements of $S$ are an abelian group with $+$ because they are a subgroup of ($\R, +$), which is an abelian group. (recall subgroup of abelian group is abelian from 411). Elements of $S$ are associative with $\cdot$ because they are in $\R$. The additive identity of reals, which $S$ must inherit, is in $S$, where $a = b = 0 \in \Z$.

For distributivity:

\begin{align*}
    (a + b\sqrt2)\left[(a' + b'\sqrt2) + (a'' + b''\sqrt2)\right]\\
    (a + b\sqrt2)\left[(a'+a'') + (b'+b''\sqrt2)\right]\\
    (aa' + aa'' + 2bb' + 2bb'') + (ab'+ab'' + a'b + a''b)\sqrt2\\
    (aa' + 2bb' ) + (ab' + a'b )\sqrt2 + (aa'' + 2bb'')+(ab''+ a''b)\sqrt2\\
    (a + b\sqrt2)(a' + b'\sqrt2) + (a + b\sqrt2)(a''+ b''\sqrt2)\\
\end{align*}

Right distributivity holds since all elements in $S$ commute in $R$, therefore they commute in $S$.

(b) $S$ isn't closed under multiplication. If $f$ and $g$ are two functions with zero 3rd derivatives, the third derivative of $f(x)g(x)$ isn't necessarily zero because of the chain rule:

\begin{align*}
    (f(x)&g(x))'''\\
    (f'(x)g(x) &+ f(x)g'(x))''\\
    (f''(x)g(x) + 2f'(x)&g'(x)+ f(x)g''(x))'\\
    f'''(x)g(x)+ 3f''(x)g'(x) &+ 3f'(x)g''(x) + f(x)g'''(x)\\
\end{align*}
The first and last terms are necessarily zero, but the middle two aren't necessarily (say if $f$ and $g$ are degree 2 polynomials with real coefficients), so $S$ isn't closed, and therefore can't be a ring.
\newpage\parskip=0em
\begin{mdframed}[backgroundcolor=blue!20]
\textbf{Problem 3}: Describe all units in a given ring $R$ explicitly
\begin{itemize}
    \item $R = \Z_4 \times \Z_4$
    \item $R = Mat_2(\Z_2)$
\end{itemize}
\end{mdframed}
\textbf{Solution}\parskip=1em\\
(a) the unity of $Z_4$ is $(1, 1)$, because 1 is the multiplicative identity of $\Z_4$. Therefore, the units of $R$ are the pairs with entries that are invertible in $\Z_4$, more specifically 1 and 3. There does not exist any element $x$ such that $2x = 1 \in \Z_4$, because $\gcd(2, 4) = 2$, so the Diophantine equation equivalent to this congruence cannot equal any positive number strictly less than 2. Therefore, the units in $R$ are $(1, 1), (1, 3), (3, 1), $ and $(3, 3)$.

(b) First note that ring has unity $\begin{pmatrix}
    1 & 0 \\
    0 & 1\\
\end{pmatrix}$. For a matrix in this ring to be a unit, it must be invertible in the traditional sense of matricies under multiplication. Meaning for a matrix $\begin{pmatrix}
    a & b \\
    c & d\\
\end{pmatrix}$, we have $ad - bc \neq 0$. In this case, the only other option is that $ad - bc = 1$. This equivalent to saying that $ad \neq bc$. All such matricies are units. 

\newpage\parskip=0em
\begin{mdframed}[backgroundcolor=blue!20]
\textbf{Problem 4}: Given an example of a ring with unit $1 \neq 0$ that has a subtring with a non-zero unity $e \neq 1$
\end{mdframed}
\textbf{Solution}\parskip=1em\\
Take $R = \Z_2 \times \Z_2$ with subring of $\Z_2 \times {0}$. The unit of the first ring is $(1, 1)$, and the unit of the second is $(1, 0)$.
\newpage\parskip=0em
\begin{mdframed}[backgroundcolor=blue!20]
\textbf{Problem 5}: Let $U$ be a collection of all units in a ring $(R, + ,\cdot)$ with unity, Prove that $(U, \cdot)$ is a group
\end{mdframed}
\textbf{Solution}\parskip=1em\\
Associative: $(R, +, \cdots)$ being a ring $\implies \cdot$ is associative for all elements in $R$. Therefore, because all elements in $U$ are also in $R$, they must all satisfy associativity under multiplication.

Identity: Say $R$ has unity $1$. $1 \in U$ because $\forall a\in R: a\cdot 1 = a \implies 1 \cdot 1 = 1$, which is the definition of a unit. Hence, $1 \in U$. Because all other units in $U$ are also in $R$, the above property of unity (or identity for groups) is satisfied, and $1 \in R$ is the identity element of $U$. 

Inverses: If $a$ is a unit in $R$, then $\exists a': aa' = 1$. Likewise, $a'$ will be a unit because $\exists a'': a'a'' = 1$, where $a'' = a$, so all units in $R$ will "bring their inverses with them" into $U$.

Closure: for two units $a, b$ with inverses $a', b'$, the product $ab$ is a unit because $abb'a = a1a' = aa' = 1$.
\newpage\parskip=0em
\begin{mdframed}[backgroundcolor=blue!20]
\textbf{Problem 6}: Let $X$ be the collection of all rings. Prove that isomorphism of rings gives an equivalence relation on $X$
\end{mdframed}
\textbf{Solution}\parskip=1em\\
Reflexivity: all $R \in X$ are isomorphic to themselves, using the trivial isomorphism $\phi: R\to R$ defined by $\phi(r) = r$ for all $r\in R$.

Symmetry: For two rings $R$ and $S$ with isomorphism $\phi: R\to S$, there exists an inverse function $\phi^{-1}: S \to R$ since ring isomorphism is bijective, which is also a bijective homomorphism (and therefore an isomorphism). For arbitrary $r, r' \in R$, where $\phi(r) = s$ and $\phi(r') = s'$, $\phi$ is a homomorphism as: $\phi^{-1}(s + s') =\phi^{-1}(\phi(r) + \phi(r')) = \phi^{-1}(\phi(r + r')) = r + r' = \phi^{-1}(s) + \phi^{-1}(s')$\\
and \\
$\phi^{-1}(ss') =\phi^{-1}(\phi(r)\phi(r')) = \phi^{-1}(\phi(rr')) = rr' = \phi^{-1}(s)\phi^{-1}(s')$. 

Transitivity: Assume that for rings $R, S, T$, there exist isomorphisms $\phi: R \to S$ and $\psi: S \to T$. Then $\psi \circ \phi: R \to T$ is an isomorphism. It is a bijection because both composing functions are bijective and the codomain of the interior function is the domain of the exterior composing function. $\psi \circ \phi$ is a homomorphism:\\ $\psi \circ \phi(r + r') = \psi(\phi(r) + \phi(r')) = \psi(s + s') = \psi(s) + \psi(s') = \psi \circ \phi(r) + \psi \circ \phi(r')$\\
and \\
$\psi \circ \phi(rr') = \psi(\phi(r)\phi(r')) = \psi(ss') = \psi(s)\psi(s') = \psi \circ \phi(r)\cdot \psi \circ \phi(r')$.
\newpage\parskip=0em
\begin{mdframed}[backgroundcolor=blue!20]
\textbf{Problem 7}: An element $x$ of a ring $R$ is called nilpotent if $x^n = 0$ for some $n>0.$
\begin{itemize}
    \item[(a)] Find all nilpotents in $\Z_{2022}$
    \item[(b)] Give an example of a ring with 2 nilpotents
    \item[(c)] Let $R$ be a commutative ring with nilpotents $x, y$. Show that $x +y$ is also nilpotent
\end{itemize}
\end{mdframed}
\textbf{Solution}\parskip=1em\\
(a) Recall that $n = rs, r, s$ coprime, $\Z_{n} \cong \Z_r \times \Z_s \implies \Z_{2022} \cong \Z_2 \times \Z_3 \times Z_{337}$. If we want an element $g^n \in \Z_{2022}$ to be 0, then all of entries in the the corresponding entries must also be 0 when raised to $n$. Because all of the rings in this example have prime order, and $g^n = 0 n>0\implies g$ is not a unit, the only possible nilpotent is $(0, 0, 0)$, or $0 \in \Z_{2022}$.

(b) $\Z_4$ has two nilpotents: 0 and 2. 1 and 3 have powers that cycle through 1, and 1 and 3 respectively.

(c) Let $x^n = 0$ and $y^m = 0$. Assume without loss of generality that $n \geq m$. Notice that if $a^b = 0 \in \Z_k$, then any greater power will also be 0, since $\gcd(a^b, k) = k$. Therefore, taking $(x + y)^{m+n}$, and expanding with binomial theorem (\emph{it is important that the ring is commutative to allow the terms to be rearranged so that it's power of x times power of y, or else this doesn't necessarily work}):
\begin{align*}
    \sum\limits_{k\geq 0} {n\choose k}x^{m+n-k}y^{k}
\end{align*}
When $k \leq m$, $m + n -k \geq n \implies x^{m + n-k} = 0$, and $k > m \implies y^k = 0$, s0 all terms in the summation are $0 \in \Z_k$. Therefore the evaluated summation of $(x + y)^{m+n} = 0 \in \Z_k$, and $x + y$ is also nilpotent
\newpage\parskip=0em
\begin{mdframed}[backgroundcolor=blue!20]
\textbf{Problem 8}: Find all solutions of the equation $x^2 + 2x + 2 = 0$ in a given ring $R$
\begin{itemize}
    \item[(a)] $R = \Z_5$
    \item[(b)] $R = \Z_7$
    \item[(c)] $R = \Z_8$
\end{itemize}
\end{mdframed}
\textbf{Solution}\parskip=1em\\
(a) $x = 1 \implies 1 + 2 + 2 = 5 = 0 \in \Z_5$, and $x = 2 \implies 4 + 4 + 2 = 10 = 0 \in \Z_5$. (there cannot be more by Lagrange's Theorem for polynomials, or you can brute force check it)

(b) No solutions. I couldn't think of a convincing argument that is less work than just checking all of them manually:\\
$0^2 + 2(0) + 2 = 2$\\
$1^2 + 2(1) + 2 = 5$\\
$2^2 + 2(2) + 2 = 3$\\
$3^2 + 2(3) + 2 = 3$\\
$4^2 + 2(4) + 2 = 5$\\
$5^2 + 2(5) + 2 = 2$\\
$6^2 + 2(6) + 2 = 1$ 

(c) No solutions. If $x = 2n$ even, $4n^2 + 4n \equiv 6 \pmod{8}$ can be reduced to $2n^2 + 2n \equiv 3\pmod{4}$. The equivalent Diophantine equation cannot be solved, because the left side will always be a multiple of 2, and therefore cannot be 3.

If $x = 2n + 1$ odd, $4n^2 + 8n + 5 \equiv 4n^2 + 5 \equiv 0 \pmod{8} \implies 4n^2 \equiv 3\pmod{8}$, which again is unsolvable since 3 is not a multiple of 2.
\newpage\parskip=0em
\begin{mdframed}[backgroundcolor=blue!20]
\textbf{Problem 9}: Show that the characteristic of an integral domain is either 0 or a prime number $p$
\end{mdframed}
\textbf{Solution}\parskip=1em\\
If $1 = 0$ in the integral domain, the characteristic is 0. Else, assume that an integral domain $R$ has characteristic $n = ab$, where $a, b \neq 1$. Then $1 + 1 + \cdots + 1 = 0$ additions can be broken up into $a$ different disjoint sets of $b$ additions. This is equivalent to $a(1 + 1 + \cdots+ 1) = 0$, with $b$ additions. This produces producing a zero division of two elements of $R$ (if they aren't elements of $R$ we have closure problems), contradicting $R$ being an integral domain. Therefore, $n$ must not be factorable (and must be a prime $p$)

\newpage\parskip=0em
\begin{mdframed}[backgroundcolor=blue!20]
\textbf{Problem 10}: For each of the following rings $R$ decide (with proof) whether $R$ is a field and whether $R$ is an integral domain:
\begin{itemize}
    \item[(a)] $R = \Z_{2021}$
    \item[(b)] $R = \{$even integers$\}$
    \item[(c)] $R = \{$polynomials with $x$ with coeffiecients in $\R\}$ ($\R[X]$)
    \item[(d)] $R = \C$
\end{itemize}
\end{mdframed}
\textbf{Solution}\parskip=1em\\
Note that we are told all four examples are rings, so I'll just skip verifying that they are rings in each proof.

(a) Neither. $43$ does not have a multiplicative inverse since $\gcd(43, 2021) > 1$ (and therefore we can't solve their linear Diophantine equation for 1), so $\Z_{2021}$ cannot be a field. $43 \cdot 47 = 0 \in \Z_{2021}$, so $\Z_{2021}$ has zero division, and therefore cannot be an integral domain.

(b) Not a field since there are no multiplicative inverses in $2\Z$. However, for $x \in 2\Z^{\neq 0}$, there does not exist a $y\in 2\Z^{\neq 0}$ such that $xy = 0$ by the definition of non-zero integer multiplication, therefore $2\Z$ is an integral domain.

(c) $R$ is not a field because taking a polynomial with degree $n > 1$ and zero constant coefficient, and attempt to invert the polynomial at $x = 0$ will result in $0 = 1$, therefore not all non-zero polynomials with real coefficints are units, and $R$ cannot be a field. $R$ is an integral domain because evaluating an element of $R$ at some $x$ such that the result is not 0 will give a unit real number (since $\R$ is a field), meaning the product of two non-zero unit polynomials results in a non-zero real.

(d) $\C$ is a field. Multiplication of complex numbers is commutative because multiplication of reals (modulus dilation) and addition of reals (argument addition) are commutative. For a complex number $z = re^{i\theta}$ with $r > 0$, $z$ can be inverted by multiplying by $z^{-1} = r^{-1}e^{-i\theta}$ to get $zz^{-1} = rr^{-1}e^{i(\theta - \theta)} = 1$. The constaint on $r$ is necessary since 0 is not invertible in the reals. Because $\C$ is a field, it is also an integral domain
\end{document}