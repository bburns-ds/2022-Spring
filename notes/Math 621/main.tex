% ENORMOUS credit for the foundational formatting
%   of these notes goes to Patrick Lei. 
% Check out his fantastic course notes
%   here: https://git.snopyta.org/abstract/course-notes
% and his website 
%   here: https://www.math.columbia.edu/~plei/index.html

\documentclass[twoside, 10pt]{article}
\usepackage{bbnotes}
\geometry{margin=2cm}

\newcommand{\D}{\mathbb{D}}
\renewcommand{\Re}{\text{Re}}

\definecolor{darkblue}{RGB}{0,0,128}
\definecolor{darkred}{RGB}{128,0,0}
\definecolor{darkyellow}{RGB}{96,96,0}
\definecolor{darkgreen}{RGB}{0,128,0}
\definecolor{darkdarkred}{RGB}{64,0,0}

\lstset
{
basicstyle=\ttfamily\scriptsize, 
breaklines=true,
postbreak=\mbox{\textcolor{darkdarkred}{$\hookrightarrow$}\space},
showstringspaces=false
keywordstyle = [1]\bfseries\color{darkred},
keywordstyle = [2]\itshape\color{darkgreen},
keywordstyle = [3]\sffamily\color{darkblue},
keywordstyle = [4]\color{darkyellow},
}

\fancypagestyle{firstpage}
{
   \fancyhf{}
   \fancyfoot[R]{\itshape Page \thepage\ of \pageref{LastPage}}
   \renewcommand{\headrulewidth}{0pt}
}


\fancypagestyle{pages}
{
\fancyhead[C]{\scshape Math 621}
\fancyhead[L]{\scshape Ben Burns}
\fancyhead[R]{\scshape Spring 2022}
\renewcommand{\headrulewidth}{0.1pt}
}

\pagestyle{pages}

\title{Math 621}
\author{Taught by Jenia Tevelev \ Scribed by Ben Burns}
\affil{UMass Amherst}
\date{Spring 2022}

\begin{document}

\maketitle\thispagestyle{firstpage}
 
\tableofcontents

\section{Singularities}
\begin{thm}[Riemann's Theorem on removeable singularities]
   $f(z)$ has an isolated singularity at $z_0$, bounded in punctured neighborhood of $z_0$, then $f$ can be exteneded to a holomorphic function at $z_0$.
\end{thm}

\begin{lem}
   If $F(s, z)$ is continuous in $[0, 1]\times \Omega$, such that $\Omega \subset \C$ open, and holomorphic in $z$, then $\int_0^1 F(s, z)ds$ is holomorphic in $\Omega$
\end{lem}
One argument: write integral as a limit of Riemann sums (obviously holomorphic in $z$) %Todo: why??
$\implies$ the limit is holomorphic if we can show uniform convergence on compact subsets of $\Omega$

Another argument: \\
(1) $\int_0^1 F(s, z)dz$ continuous in $\Omega$. Choose $\overline{\D}$ such that $z \in \overline{\D} \subset \Omega$. $F(s, z)$ is uniformly continuous on $[0, 1]\times \Omega$. $\forall \epsilon \exists \delta$ such that if $|z' - z| < \delta$ then $|F(s, z') - F(s, z)| < \epsilon$ $\forall s \in [0, 1] \implies \left|\int_0^1 F(s, z') - F(s, z) dz\right| \leq \int_0^1|F(s, z') - F(s, z)| ds < \int_0^1dz \epsilon = \epsilon - 0 = \epsilon \implies \int_0^1 F(s, z)ds$ continuous in $z$.

(2) By Morera's Theorem, it suffices to check $\int_T \left(\int_0^1 F(s, z)ds\right) dz = 0, T \subset\Omega$ with its interior. By Frobinius %Todo: review
$= \int_0^1\left(\int_T F(s, z) dz\right) ds = 0$. Interior integral is zero by Cauchy or Goursat, so entire integral is 0, and $f$ is holomorphic by Morera's.

\begin{cor}
   $z_0$ is a pole of $f(z) \iff \lim\limits_{z \to z_0}|f(z)| = \infty$
\end{cor}
$z_0$ is a pole of $f(z) \implies z_0$ is a zero of $\dfrac{1}{f(z)} \implies \lim\limits_{z\to z_0}\dfrac{1}{f(z)} = 0 \implies \lim\limits_{z\to z_0}|f(z)|= \infty$. \\ 
If $\lim\limits_{z\to z_0}|f(z)|= \infty \implies \lim\limits_{z\to z_0}\left| \dfrac{1}{f(z)} = 0\right| \implies \dfrac{1}{f(z)}$ is bounded near $z_0 \implies \dfrac{1}{f(z)}$ can be extended to a holomorphic function at $z_0$, all if $g(z), g(z_0) = \lim\limits_{z\to z_0}g(z) = 0 \implies f(z) = \dfrac{1}{g(z)}$ has a pole at $z_0$.

\begin{defn}
   Refined classification of isolated singularities at $z_0$.
\end{defn}
(1) removable $\iff f(z)$ is bounded near $z_0 \iff f(z)$ is holomorphic at $z_0$ (last by Riemann)

(2) pole $\iff f(z) = \dfrac{1}{g(z)}, g(z)$ is holomorphic, $g(z_0) = 0 \iff \lim\limits_{z\to z_0}|f(z)| = \infty$

(3) Essential singularities? (holo on deleted neighborhood but not remov sing. or pole)

\begin{exm}
   $e^{1/z}$ at $z=0$
\end{exm}

\begin{thm}[Casorati-Weierstrass]
   $z_0$ is an essential singularity of $f(z) \implies f(0 < |z-z_0| < r)$ dense in $\C \forall r$. 
\end{thm}

Argue by contradiction: suppose $\exists w_0$ and $R$ such that $|f(z) - w_0| > R \forall z$ such that $0 < |z-z_0| < r$ %Todo: review annulus (nested triangle thing?)
$\dfrac{1}{f(z)-w_0} < R$ is bounded in same annulus. So by Riemann's Theorem, $=g(z)$, which is holomorphic for $|z-z_0| < r$ (by Riemann's Theorem). So $f(z) = w_0 + \dfrac{1}{g(z)}$. If $g(z_0) \neq 0$, then $f$ is holomorphic at $z_0$. If $g(z_0) = 0 \implies w_0 + \dfrac{1}{g(z)}$ has a pole at $z_0$, which is a contradiction %Todo: why?

\begin{thm}[Picard's Theorem]
   Every $\alpha \in \C$, with at most one exception, belongs to the image $f(0 < |z-z_0| < r) \forall r$, and occurs infinitely many times
\end{thm}
Covers all for given $r$, shrink $r$, covers all by Picard's, shrink, etc.

"Singularity at $\infty$": in book

"Riemann Sphere" $S^2 = \C \P^1 = \C$ disjoint union $\{\infty\}$.

$f(z)$ has an isolated singularity at $\infty \iff f(\dfrac{1}{z})$ is holomorphic for $0 < \left|\dfrac{1}{z}\right| < \dfrac{1}{R} \iff f(z)$ is holomorphic for $|z| > R$ for some $R$.

$f(z)$ has a removable singularity at $\infty \iff f(z)$ is bounded for $|z| > R \iff f(\dfrac{1}{z})$ is holomorphic at $z=0 \iff f(\dfrac{1}{z}) = \sum\limits_{n\geq 0}a_n\left(\dfrac{1}{z}\right)^n$ converges for $\dfrac{1}{z} < r$

$f(z)$ has a pole at $\infty \iff f(\omega)$ has a pole at $0, \omega = 1/z \iff \lim\limits{\omega\to0}|f(\omega)| = \infty \iff\lim\limits_{z\to\infty}|f(z)| = \infty$.

$f(\omega) = \dfrac{a_{-n}}{\omega^n} + \ldots + \dfrac{a_{-1}}{\omega} + H(\omega)$ holo at $\omega \iff$ bounded near 0. $f(z) = a_{-n}z^n + \ldots + a_{-1}z + H(\dfrac{1}{z})$ bounded at $\infty$ (for $|z| > R$ for some $R$)

Theorem 3.4.

argument principle

\begin{thm}[Roche Theorem]
   Suppose $f(z)$ and $g(z)$ are holomorphic in $\Omega$, which contains a simple closed curve $\gamma$ and its interior. Supppose $|f(z)| > |g(z)| \forall z \in \gamma$. Then $f(z)$ and $f(z) + g(z)$ have the same number of zeros (counted with multiplication) inside $\gamma$.
\end{thm}
Let $f_s(z) = f(z) + sg(z), s\in [0, 1]$. $f_s(z)$ is holomorphic in $\Omega \forall s \in [0, 1]$. $\forall z\in \gamma$ $|f_s(z)| = |f(z) + sg(z)| \geq |f(z) - s|g(z)| \geq 0 \implies f_s(z)$ doesn't vanish along $\gamma \implies $ the number of zeros of $f_s(z)$ inside $\gamma$ (with multiplication) is equal to $= \dfrac{1}{2\pi}\int_\gamma \dfrac{f'(z) + sg'(z)}{f(z) + sg(z)}dz$.
 The integral is a continuous function of $(z, s)$ on $\gamma \times [0, 1]$ (compact) %Todo: why?
$\implies \dfrac{1}{2\pi}\int_\gamma \dfrac{f'(z) + sg'(z)}{f(z) + sg(z)}dz$ is continuous in $S$. But it takes integer values $\implies \dfrac{1}{2\pi}\int_\gamma \dfrac{f'(z) + sg'(z)}{f(z) + sg(z)}dz$ is constant in $S$. In particular, the same for $f(z)$ (where $s = 0$)and $f(z) + g(z)$ (where $s = 1$)

\begin{exm}
   Find the number of zeros inside $|z| < 1$ of $z^{100} + 4z^3 -z + 1 = f(z) + g(z)$. $f(z) = 4x^3$, with 3 zeros with multiplication, and $g(x) = z^{100} - z + 1$. $|4z^3| = 4$. $|g(x)| \leq 3 < 4$.
\end{exm}
%Todo: gcd of polynomial and derivative to find multiplicity of roots 

\begin{thm}[Open Mapping Theorem]
   If $f(z)$ is holomorphic in connected open $\Omega$ and nonconstant $\implies f: \Omega \to f(\Omega)$ is an open map (sends open sets to open sets).
\end{thm}
\begin{rmk}
   We want connected to avoid something like HW1P4
\end{rmk}
It suffices to show that $f(\Omega)$ is open. Say $f(z_0) = \omega_0$. Show the image of $f$ contains some neighborhood of $\omega_0$, or $\exists r > 0$ such that $\{|\omega-\omega_0| < r\} \subset f(\Omega)$. Equivalent to sayint that $f(z) - \omega$ has a root in $\Omega$ $\forall \omega$ such that $|\omega - \omega_0| < r$. $f(z) - \omega$ has a solution $z_0$. Chose a circle $|z - z_0| = \delta$. $f(z) - w_0$ has a zero inside the circle.
Apply Rouche theorem. 

We know $\exists \delta$ such that $f(z) - \omega_0 \neq 0$ for some $z, |z-z_0| = \delta$ because roots form discrete set. Take $r = \min|f(z) - \omega_0|$, chain inequaliities, apply Rouche Theorem %Why does this say $f(z) = w$.

%Why can't an open set have a maximum holomorphic?

\section{Riemann Zeta function}
Say $s > 1$, and the sum $\sum\dfrac{1}{n^s} = \zeta(s)$ converges for $\forall s > 1$. In Calculus, you prove this converges with the integral test.

\begin{thm}
   $\zeta(s)$ admits an analytic continutation to $\C \setminus 1$ where it has a simple pole
\end{thm}

\begin{thm}
   The only zeros of $\zeta(s)$ outside the "critical strip" $0 \leq \Re(s) \leq 1$ are "trivial zeros": -2, -4,$\ldots$. (simple zeros of Gamma function)
\end{thm}

\begin{thm}[Hadamard, Valle Poussin]
   No zeros exist on the boundary of the critial strip
\end{thm}

\begin{thm}[Riemann Hypothesis]
   All zeros are in the middle of the strip, along $\Re(s) = 1/2$.
\end{thm}

\begin{defn}[Tchebyshev's $\psi$-function]
   $\psi(x) = \sum\limits_{p^m \leq X} \log p$, $p$ is prime, $x \in \R, x > 0$.
\end{defn}

\begin{defn}[Riemann's Explicit Formula]
   $\psi(x) = x - \sum\limits_{\zeta(\rho) = 0}\dfrac{x^\rho}{\rho} - \ln(2\pi) - \dfrac{1}{2}\ln(1-x^{-2})$, $0 < \Re(\rho) < 1$
\end{defn}

\begin{defn}[Wacky Prime Number Theorem]
   As $x\to\infty$, the last terms just goes away. $\psi(x) \sim X$, or $\lim\limits_{x\to\infty} \dfrac{\psi(x)}{x} = 1$
\end{defn}

If you believe the Riemann hypothesis, $\rho = 1/2$

$\zeta(s) = \sum \dfrac{1}{n^s}, \Re(s) > 1$

\begin{defn}[Euler's product expansion]
   $\zeta(s) = \prod\limits_{p} \dfrac{1}{1-{\dfrac{1}{p^s}}}, \Re(s) > 1$ for $p$ prime
\end{defn}

\begin{cor}
   $\zeta(s)$ has no zeros in $\Re(s) > 1$
\end{cor}

Idea: $\dfrac{1}{1-\dfrac{1}{p^s}} = 1 + \dfrac{1}{p^s} + \dfrac{1}{p^{2s}} + \ldots$

$\implies \prod\limits_{p} \dfrac{1}{1-{\dfrac{1}{p^s}}} = \sum \dfrac{1}{p_{i_1}^{k_{i_1} s}\cdots p_{i_r}^{k_{i_r}}} = \sum\limits_{n \geq 1} \dfrac{a(n)}{n^s}$ where $a(n)$ is the number of ways to write $n$

Function is holomorphic unless $e^{s\ln p} = 1$, but can't happen since $\Re(s) > 1$. Also, cannot vanish. Product of non-vanishing factors can't vanish (?)

Caution: $\dfrac{1}{2} \cdot\dfrac{1}{2} \cdot\dfrac{1}{2} \cdots \to 0$, so need to justify

$\prod\limits_{p < N} (1 + \dfrac{1}{p^s} + \ldots + \dfrac{1}{p^{Ms}}) < 1 + \dfrac{1}{2^s} + \ldots = \zeta(s)$\\
send $M$ to infinity, holds, send $N$ to infinity, holds. %Todo: ask about zeta originally

\textbf{Claim:} $\prod\limits_{p} \dfrac{1}{1-{\dfrac{1}{p^s}}}$ is holomorphic for $\Re(s) > 1$

\begin{prop}
   Suppose $F_n(z)$ holom. in $\Omega$, $|F_n(z)| \leq c_n$ in $\Omega$ and x $\sum c_n < \infty$. Then $\prod_{n \geq 1} (1 + F_n(z))$ converges uniformly in $\Omega$ to a holomorphic function in $\Omega$, which is equal to 0 only at zeros of its factors.
\end{prop}
As the limit of a sequence of holom. functions converges uniformly on compact sets %Todo: why

In our case: need to show that $\prod (1 - \dfrac{1}{p^s})$ is holomorphic. \\
$\left|\dfrac{1}{p^s}\right| = \dfrac{1}{p^{\Re(s)}}$.\\
$\sum \dfrac{1}{p^\Re(s)} < \sum \dfrac{1}{n^{\Re(s)}} < \infty$ since $\Re(s) > 1$\\
$\implies  \prod(1 = \dfrac{1}{p^s})$ is holomorphic for $\Re(s) < 1$\\
doesn't vanish $\implies \prod \dfrac{1}{1-\dfrac{1}{p^s}}$ is holomorphic for $\Re(s) > 1$

For $n >> 0$, $|F_n(z)| < \dfrac{1}{2}$.
Let's assume that this is true $\forall n$\\
$\implies \Re(1 + F_n(z)) > 0$\\
$\implies \log(1 + F_n(z))$ is defined, where log is standard branch.

Let's show that $\sum \log(1 + F_n(z))$ converges absolutely and uniformly

$| \log(1 + w)|$ for $|w| < \dfrac{1}{2}$\\
$\log(1 + w) = w - \dfrac{w^2}{2} + \dfrac{w^3}{3} - \cdots$\\
$\implies \log(1 + w) = w [1- \dfrac{w}{2} + \dfrac{w^2}{3} - \cdots ]$\\
$\implies |\log(1 + w)| \leq |w| [1+ \dfrac{|w|}{2} + \dfrac{|w|^2}{3} + \cdots ]$\\
$\implies |\log(1 + w)| \leq |w| [1+ \dfrac{|w|}{2} + \dfrac{|w|^2}{3} + \cdots ]$\\
$\implies |\log(1 + w)| \leq |w| [1+ |w| + |w|^2 + \cdots ]$\\ 
$\implies |\log(1 + w)| \leq 2|w|$

$\prod \dfrac{1}{1-\dfrac{1}{p}} = \infty$ because $ \leq \sum \dfrac{1}{n} = \infty$
$\implies \prod (1 - \dfrac{1}{p}) 0$

Why doesn't this contradict the previous theorem?

$\implies \sum \dfrac{1}{p} = \infty$ (b.c otherwise the previous theorem applies as the terms are no zeros)\\
$\implies$ there are infinitely many primes

\begin{thm}
   $\zeta(s)$ has no zeros for $\Re(s) = 1$ (assuming $\zeta(s)$ has an analytic continutation into $\Re(s) > 0$ with simple pole at $s = 1$ and no other poles)
\end{thm}

Argue by contradiction, suppose that $\zeta(1 + it) = 0$

\begin{rmk}
   $|\zeta^3(s)\zeta^4(\sigma + it)\zeta(\sigma + 2it)| \to 0$ as $\sigma \to 1^+$
\end{rmk}

\begin{rmk}
   $|\zeta^3(s)\zeta^4(\sigma + it)\zeta(\sigma + 2it)| \geq 1$ for $\sigma > 1$
\end{rmk}

Past two remarks give the proof

\begin{rmk}
   $\zeta(s)$ has a simple pole at $s = 1 \implies \zeta(s) = \dfrac{1}{s-1} +$ holomorphic near $s = 1$\\
   $\implies |\zeta^3(\sigma)| = \dfrac{|a|^3}{|\sigma -1|^3} +$ bounded near $\sigma = 1$
\end{rmk}

Last time: if $\zeta(1 + it) = 0 \implies |\zeta^3(s)\zeta^4(\sigma + it)\zeta(\sigma + 2it)| \to 0$ as $\sigma \to 1^+$ assuming $\zeta(s)$ has an analytic continutation to $\Re(s) > 0$ with a simple pole at 1 and no other poles along $\Re(s) = 1$

$|\zeta^3(s)\zeta^4(\sigma + it)\zeta(\sigma + 2it)| \geq 1$ for $\sigma > 1$

$3\ln|\zeta(s)| + 4\ln|\zeta^4(\sigma + it)| + \ln|\zeta(\sigma + 2it)| \geq 0$ (regular natural logarithm for real numbers)

$\Re(1) > 1 \implies \zeta(s) = \prod\limits_{\text{primes}} \dfrac{1}{1 - p^{-s}}$, take absolute value of both sides, take logarithm is $\sum\limits_{\text{primes}}-ln|1 - p^{-s}|$

$|p^{-s}| < 1 \implies$ we can compute principal branch $\log(1 - p^{-s}) \implies -\ln|1-p^{-s}| = Re - \log(1 - p^{-s})$

$|z| < 1 \implies -log(1 - z) = z + \dfrac{z^2}{2} + \dfrac{z^3}{3} + \ldots$
$\implies -\log(1 - p^{-s}) = p^{-s} + \dfrac{p^{-2s}}{2} + \ldots$

$\Re(\sum$ primes ...) [aka $\ln|\zeta(s)$]= $\sum\limits_{n \geq 1} \dfrac{c_n}{n^s} c_n \geq 0$

\subsection{Schwarz function} idk where this is going

$f(x) \in C^\infty (\R)$ is a Schwarz function if $f$ and all its derivatives decay faster than any 1/polynomial function.

$| \dfrac{\partial f}{\partial x^m} x^n < c_{n, m}|$ for some $c_{n, m}$ Basic example: $e^{-\pi x^2}$. At some point we will assume that $f(x)$ is also even.

\begin{defn}
   An associated $\theta$ function $\theta_f(y) = \sum\limits_{n \in \Z}f(y \cdot n)$ $y > 0$
\end{defn}
$C^\infty$ function

\begin{defn}
   Asn associated gamma function $\Gamma_f(s) = \int_0^\infty t^s f(t) \dfrac{dt}{t}$.
\end{defn}
This function will be analytic if $\Re(s) > 0$

If $f(x) = e^{-\pi x^2}$ then $\theta_f(y) = \sum\limits_{n \in Z} e^{-\pi n^2 y^2}$

Recall Jacobi theta function $\theta(z) = \sum\limits_{n \in \Z} e^{\pi i n^2 z}$, $z\in \C$.

$\theta_f(y) = \theta(iy^2)$

"Gamma factor" $\Gamma_f(s) = \int_0^\infty t^se^{\pi t^2} \dfrac{dt}{t}$\\
$x = \pi t^2$, $ = x^{1/2}\pi^{-1/2}, dx/x = 2\pi t dt/(\pi t^2) = 2dt/t$\\
$\implies \int_0^\infty x^{s/2}\pi^{-s/2}e^{-x}$\\
$ = \dfrac{1}{2}\pi^{-s/2}\int_0^\infty x^{s/2}e^{-x} \dfrac{dx}{x} = \dfrac{1}{2}\pi^{s/2} \Gamma(s/2)$

\begin{lem}
   $\theta_f(y) \in C^\infty(\R)$ and derivatives can be computed term by term. 
\end{lem}
Just need to show differentiability ($f'$ is also Schwarz)

Check uniform and absolute convergence (on compact sets)

$|f(yn)| \leq \dfrac{C}{(yn)^2} = \dfrac{C}{y^2n^2}$

$\sum |f(yn)| \leq \dfrac{C}{y^2}\sum\dfrac{1}{n^2}$. Absolute convergence

$\left| \int_a^b f(t, z) dt\right| \underset{a, b \to \infty} {\to}0$ uniformly for $z \in K \subset \Omega$

Basic case: prove absolute convergence along with uniform convergence. Need bound $|f(t, z)| \leq g(t)$ continuous for $z\in K$ and $\int_1^{\infty} g(t) dt < \infty$.

Given $g(t)$, $\left|\int_a^b f(t, z) dt\right| \leq \int_a^b g(t) dt \underset{a, b \to \infty}{\to} 0$

Warning: one of the hw problems, convergence of improper integral is uniform but not absolute. Play with $\int_a^b f(t, z) dt$ before 

\end{document}