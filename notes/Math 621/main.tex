% ENORMOUS credit for the foundational formatting
%   of these notes goes to Patrick Lei. 
% Check out his fantastic course notes
%   here: https://git.snopyta.org/abstract/course-notes
% and his website 
%   here: https://www.math.columbia.edu/~plei/index.html

\documentclass[twoside, 10pt]{article}
\usepackage{bbnotes}
\geometry{margin=2cm}

\newcommand{\D}{\mathbb{D}}

\definecolor{darkblue}{RGB}{0,0,128}
\definecolor{darkred}{RGB}{128,0,0}
\definecolor{darkyellow}{RGB}{96,96,0}
\definecolor{darkgreen}{RGB}{0,128,0}
\definecolor{darkdarkred}{RGB}{64,0,0}

\lstset
{
basicstyle=\ttfamily\scriptsize, 
breaklines=true,
postbreak=\mbox{\textcolor{darkdarkred}{$\hookrightarrow$}\space},
showstringspaces=false
keywordstyle = [1]\bfseries\color{darkred},
keywordstyle = [2]\itshape\color{darkgreen},
keywordstyle = [3]\sffamily\color{darkblue},
keywordstyle = [4]\color{darkyellow},
}

\fancypagestyle{firstpage}
{
   \fancyhf{}
   \fancyfoot[R]{\itshape Page \thepage\ of \pageref{LastPage}}
   \renewcommand{\headrulewidth}{0pt}
}


\fancypagestyle{pages}
{
\fancyhead[C]{\scshape Math 621}
\fancyhead[L]{\scshape Ben Burns}
\fancyhead[R]{\scshape Spring 2022}
\renewcommand{\headrulewidth}{0.1pt}
}

\pagestyle{pages}

\title{Math 621}
\author{Taught by Jenia Tevelev \ Scribed by Ben Burns}
\affil{UMass Amherst}
\date{Spring 2022}

\begin{document}

% \maketitle\thispagestyle{firstpage}
% 
% \tableofcontents

\section{Singularities}
\begin{thm}[Riemann's Theorem on removeable singularities]
   $f(z)$ has an isolated singularity at $z_0$, bounded in punctured neighborhood of $z_0$, then $f$ can be exteneded to a holomorphic function at $z_0$.
\end{thm}

\begin{lem}
   If $F(s, z)$ is continuous in $[0, 1]\times \Omega$, such that $\Omega \subset \C$ open, and holomorphic in $z$, then $\int_0^1 F(s, z)ds$ is holomorphic in $\Omega$
\end{lem}
One argument: write integral as a limit of Riemann sums (obviously holomorphic in $z$) %Todo: why??
$\implies$ the limit is holomorphic if we can show uniform convergence on compact subsets of $\Omega$

Another argument: \\
(1) $\int_0^1 F(s, z)dz$ continuous in $\Omega$. Choose $\overline{\D}$ such that $z \in \overline{\D} \subset \Omega$. $F(s, z)$ is uniformly continuous on $[0, 1]\times \Omega$. $\forall \epsilon \exists \delta$ such that if $|z' - z| < \delta$ then $|F(s, z') - F(s, z)| < \epsilon$ $\forall s \in [0, 1] \implies \left|\int_0^1 F(s, z') - F(s, z) dz\right| \leq \int_0^1|F(s, z') - F(s, z)| ds < \int_0^1dz \epsilon = \epsilon - 0 = \epsilon \implies \int_0^1 F(s, z)ds$ continuous in $z$.

(2) By Morera's Theorem, it suffices to check $\int_T \left(\int_0^1 F(s, z)ds\right) dz = 0, T \subset\Omega$ with its interior. By Frobinius %Todo: review
$= \int_0^1\left(\int_T F(s, z) dz\right) ds = 0$. Interior integral is zero by Cauchy or Goursat, so entire integral is 0, and $f$ is holomorphic by Morera's.

\begin{cor}
   $z_0$ is a pole of $f(z) \iff \lim\limits_{z \to z_0}|f(z)| = \infty$
\end{cor}
$z_0$ is a pole of $f(z) \implies z_0$ is a zero of $\dfrac{1}{f(z)} \implies \lim\limits_{z\to z_0}\dfrac{1}{f(z)} = 0 \implies \lim\limits_{z\to z_0}|f(z)|= \infty$. \\ 
If $\lim\limits_{z\to z_0}|f(z)|= \infty \implies \lim\limits_{z\to z_0}\left| \dfrac{1}{f(z)} = 0\right| \implies \dfrac{1}{f(z)}$ is bounded near $z_0 \implies \dfrac{1}{f(z)}$ can be extended to a holomorphic function at $z_0$, all if $g(z), g(z_0) = \lim\limits_{z\to z_0}g(z) = 0 \implies f(z) = \dfrac{1}{g(z)}$ has a pole at $z_0$.

\begin{defn}
   Refined classification of isolated singularities at $z_0$.
\end{defn}
(1) removable $\iff f(z)$ is bounded near $z_0 \iff f(z)$ is holomorphic at $z_0$ (last by Riemann)

(2) pole $\iff f(z) = \dfrac{1}{g(z)}, g(z)$ is holomorphic, $g(z_0) = 0 \iff \lim\limits_{z\to z_0}|f(z)| = \infty$

(3) Essential singularities? (holo on deleted neighborhood but not remov sing. or pole)

\begin{exm}
   $e^{1/z}$ at $z=0$
\end{exm}

\begin{thm}[Casorati-Weierstrass]
   $z_0$ is an essential singularity of $f(z) \implies f(0 < |z-z_0| < r)$ dense in $\C \forall r$. 
\end{thm}

Argue by contradiction: suppose $\exists w_0$ and $R$ such that $|f(z) - w_0| > R \forall z$ such that $0 < |z-z_0| < r$ %Todo: review annulus (nested triangle thing?)
$\dfrac{1}{f(z)-w_0} < R$ is bounded in same annulus. So by Riemann's Theorem, $=g(z)$, which is holomorphic for $|z-z_0| < r$ (by Riemann's Theorem). So $f(z) = w_0 + \dfrac{1}{g(z)}$. If $g(z_0) \neq 0$, then $f$ is holomorphic at $z_0$. If $g(z_0) = 0 \implies w_0 + \dfrac{1}{g(z)}$ has a pole at $z_0$, which is a contradiction %Todo: why?

\begin{thm}[Picard's Theorem]
   Every $\alpha \in \C$, with at most one exception, belongs to the image $f(0 < |z-z_0| < r) \forall r$, and occurs infinitely many times
\end{thm}
Covers all for given $r$, shrink $r$, covers all by Picard's, shrink, etc.

"Singularity at $\infty$": in book

"Riemann Sphere" $S^2 = \C \P^1 = \C$ disjoint union $\{\infty\}$.

$f(z)$ has an isolated singularity at $\infty \iff f(\dfrac{1}{z})$ is holomorphic for $0 < \left|\dfrac{1}{z}\right| < \dfrac{1}{R} \iff f(z)$ is holomorphic for $|z| > R$ for some $R$.

$f(z)$ has a removable singularity at $\infty \iff f(z)$ is bounded for $|z| > R \iff f(\dfrac{1}{z})$ is holomorphic at $z=0 \iff f(\dfrac{1}{z}) = \sum\limits_{n\geq 0}a_n\left(\dfrac{1}{z}\right)^n$ converges for $\dfrac{1}{z} < r$

$f(z)$ has a pole at $\infty \iff f(\omega)$ has a pole at $0, \omega = 1/z \iff \lim\limits{\omega\to0}|f(\omega)| = \infty \iff\lim\limits_{z\to\infty}|f(z)| = \infty$.

$f(\omega) = \dfrac{a_{-n}}{\omega^n} + \ldots + \dfrac{a_{-1}}{\omega} + H(\omega)$ holo at $\omega \iff$ bounded near 0. $f(z) = a_{-n}z^n + \ldots + a_{-1}z + H(\dfrac{1}{z})$ bounded at $\infty$ (for $|z| > R$ for some $R$)

Theorem 3.4.

argument principle

\begin{thm}[Roche Theorem]
   Suppose $f(z)$ and $g(z)$ are holomorphic in $\Omega$, which contains a simple closed curve $\gamma$ and its interior. Supppose $|f(z)| > |g(z)| \forall z \in \gamma$. Then $f(z)$ and $f(z) + g(z)$ have the same number of zeros (counted with multiplication) inside $\gamma$.
\end{thm}
Let $f_s(z) = f(z) + sg(z), s\in [0, 1]$. $f_s(z)$ is holomorphic in $\Omega \forall s \in [0, 1]$. $\forall z\in \gamma$ $|f_s(z)| = |f(z) + sg(z)| \geq |f(z) - s|g(z)| \geq 0 \implies f_s(z)$ doesn't vanish along $\gamma \implies $ the number of zeros of $f_s(z)$ inside $\gamma$ (with multiplication) is equal to $= \dfrac{1}{2\pi}\int_\gamma \dfrac{f'(z) + sg'(z)}{f(z) + sg(z)}dz$.
 The integral is a continuous function of $(z, s)$ on $\gamma \times [0, 1]$ (compact) %Todo: why?
$\implies \dfrac{1}{2\pi}\int_\gamma \dfrac{f'(z) + sg'(z)}{f(z) + sg(z)}dz$ is continuous in $S$. But it takes integer values $\implies \dfrac{1}{2\pi}\int_\gamma \dfrac{f'(z) + sg'(z)}{f(z) + sg(z)}dz$ is constant in $S$. In particular, the same for $f(z)$ (where $s = 0$)and $f(z) + g(z)$ (where $s = 1$)

\begin{exm}
   Find the number of zeros inside $|z| < 1$ of $z^{100} + 4z^3 -z + 1 = f(z) + g(z)$. $f(z) = 4x^3$, with 3 zeros with multiplication, and $g(x) = z^{100} - z + 1$. $|4z^3| = 4$. $|g(x)| \leq 3 < 4$.
\end{exm}
%Todo: gcd of polynomial and derivative to find multiplicity of roots 

\begin{thm}[Open Mapping Theorem]
   If $f(z)$ is holomorphic in connected open $\Omega$ and nonconstant $\implies f: \Omega \to f(\Omega)$ is an open map (sends open sets to open sets).
\end{thm}
\begin{rmk}
   We want connected to avoid something like HW1P4
\end{rmk}
It suffices to show that $f(\Omega)$ is open. Say $f(z_0) = \omega_0$. Show the image of $f$ contains some neighborhood of $\omega_0$, or $\exists r > 0$ such that $\{|\omega-\omega_0| < r\} \subset f(\Omega)$. Equivalent to sayint that $f(z) - \omega$ has a root in $\Omega$ $\forall \omega$ such that $|\omega - \omega_0| < r$. $f(z) - \omega$ has a solution $z_0$. Chose a circle $|z - z_0| = \delta$. $f(z) - w_0$ has a zero inside the circle.
Apply Rouche theorem. 

We know $\exists \delta$ such that $f(z) - \omega_0 \neq 0$ for some $z, |z-z_0| = \delta$ because roots form discrete set. Take $r = \min|f(z) - \omega_0|$, chain inequaliities, apply Rouche Theorem %Why does this say $f(z) = w$.

%Why can't an open set have a maximum holomorphic?

\end{document}