% ENORMOUS credit for the foundational formatting
%   of these notes goes to Patrick Lei. 
% Check out his fantastic course notes
%   here: https://git.snopyta.org/abstract/course-notes
% and his website 
%   here: https://www.math.columbia.edu/~plei/index.html

\documentclass[twoside, 10pt]{article}
\usepackage{bbnotes}
\geometry{margin=2cm}

\definecolor{darkblue}{RGB}{0,0,128}
\definecolor{darkred}{RGB}{128,0,0}
\definecolor{darkyellow}{RGB}{96,96,0}
\definecolor{darkgreen}{RGB}{0,128,0}
\definecolor{darkdarkred}{RGB}{64,0,0}

\lstset
{
basicstyle=\ttfamily\scriptsize, 
breaklines=true,
postbreak=\mbox{\textcolor{darkdarkred}{$\hookrightarrow$}\space},
showstringspaces=false
keywordstyle = [1]\bfseries\color{darkred},
keywordstyle = [2]\itshape\color{darkgreen},
keywordstyle = [3]\sffamily\color{darkblue},
keywordstyle = [4]\color{darkyellow},
}

\fancypagestyle{firstpage}
{
   \fancyhf{}
   \fancyfoot[R]{\itshape Page \thepage\ of \pageref{LastPage}}
   \renewcommand{\headrulewidth}{0pt}
}


\fancypagestyle{pages}
{
\fancyhead[C]{\scshape Math 571}
\fancyhead[L]{\scshape Ben Burns}
\fancyhead[R]{\scshape Spring 2022}
\renewcommand{\headrulewidth}{0.1pt}
}

\pagestyle{pages}

\title{Math 571}
\author{Taught by Tom Weston \\ Scribed by Ben Burns}
\affil{UMass Amherst}
\date{Spring 2022}

\newcommand{\F}{\mathbb{F}}
\renewcommand{\O}{\mathbb{O}}
\usepackage{algorithmicx}

\begin{document}

\maketitle\thispagestyle{firstpage}

\tableofcontents

\section{Elliptic Curves}
In the 1980s, Lenstra found a way to apply the very developed theory of elliptic curves to cryptography and factorization. 

\begin{defn}
    An elliptic curve is a plane cubic curves given by an equation $y^2 = x^3 + ax + b$ with $a, b \in \Q$ s.t $\Delta = 4a^3 + 27b^2 \neq 0$
\end{defn}

\begin{rmk}
    Most general equation, the Weierstrass equation: $y^2 = a_1xy + a_3y = x^3 + a_2x^2 + a_4x + a_6$
\end{rmk}

\subsection{Point addition}
Define $E := y^2 = x^3 + ax + b$. The key thing is the addition law. Given $P, Q$ points on $E$, construct a third point $P \oplus Q$

\begin{thm}[Bezout's Theorem]
    A curve of degree $d$ and a curve of degree $d'$ have $dd'$ points of intersection
\end{thm}
Two cocentric circles won't have any intersections $\to$ requires complex numbers. 

Take elliptic curve of degree 3, and a line of degree one. By Bezout's Theorem, there will be two points of intersection. Two of which are $P$ and $Q$, and call the third $R$. Set $P \oplus Q$ to be the reflection of $R$ across the $x$-axis. With a few other conditions, we get a group law.

\begin{exm*}
    $y^2 = x^3 - 15x + 18$. $P$ = (7, 16) $Q$ = (1, 2)
\end{exm*}
$y - 2 = \dfrac{7}{3}(x-1) \implies y = \dfrac{7}{3}x - \dfrac{1}{3}$. Insert into elliptic curve $(\dfrac{7}{3}x - \dfrac{1}{3}) = x^3 - 15x + 18 \implies \dfrac{49}{9}x^2 - \dfrac{14}{9}x + \dfrac{1}{9} = x^3 - 15x + 18 \implies x^3 - \dfrac{49}{9}x^2 + \ldots = 0$. Move all terms to one side, and solve the cubic. 

Don't need the cubic equation, because we know that $P$ and $Q$ are on the intersection, or $x = 7$ and $x = 1$ are two zeros. $(x - 1)(x - 7)(x - x_0) \implies x^3 - (8 + x_0)x^2 + \ldots$, equate the quadratic coefficients $\dfrac{-49}{9}$ = $-(8 + x_0) \implies x_0 = \dfrac{-23}{9}$. Therefore $R$ has an $x$ value of $\dfrac{-23}{9}$. 

Caveats: if we take the same point twice, take the tangent line rather than a secant line. If you take two points on a vertical line, your third is the projective point at infinity.

$E(\R) = \{(x, y) \in \R^2 | y^2 = x^3 + ax + b\} \cup \{\O\}$, where $\O$ is the point at infinity.

Assuming we have the two points $P = (x_1, y_1)$ and $Q = (x_2, y_2)$, where $x_1 \neq x_2$.

1) (Secant) Line PQ\\
$Y = y_1 + \lambda(X - x_1), \lambda = \dfrac{y_2 - y}{x_2 - x}$

2) Insert into cubic\\
$(y_1 + \lambda(X - x_1))^2 = X^3 + aX + b$\\
$0 = X^3 + (-\lambda)X^2 + \ldots$

We know this must factor into $(X-x_1)(X-x_2)(X-x_3)$ since $P$ and $Q$ are on the line and on $E$.

3) Equate coefficient of $X^2$\\
$-\lambda^2 = -(x_1 + x_2 + x_3)$\\
$x^3 = \lambda^2 - x_1 - x_2$

4) Plug $X = x_3$ into line\\
$y_3 = t_1 + \lambda(x_3 - x_1)$

5) $P\oplus Q = (x_3, -y_3)$

\begin{rmk}
    This exercise is not to suggest memorizing this algorithm, just to demonstrate that there is a general solution method for two points with distinct $x$ values on $E$.
\end{rmk}

\subsection{Special Cases} Now we address more special cases of point addition

1) $\O \oplus Q = Q$, $P \oplus \O = P$.

2) $P = (x, y)$\\
$-P = (x, -y)$ (reflection across $x$-axis)\\
$P \oplus -P = \O$\\

3) $P \oplus P$: The only difference from the general case is that, here, $\lambda$ is the slope of the tangent line of $E$ at $P$, which can be determined by implicit differentiation $\implies 2YY' = 3X^2 + a \implies Y'  = \dfrac{3X + a}{2Y} \implies \lambda = \dfrac{3x_1^2 + a}{2y_1}$. 

\begin{rmk}
    In this 3rd case, if $y_1$ is zero, this obviously doesn't work. However, that is just where $P$ is on the $x$-axis, and is therefore its own reflection, so $P \oplus P = P \oplus -P = \O$
\end{rmk}

\begin{prop}
    $E(\R) = \{(x, y) \in \R^2 | y^2 = x^3 + ax + b\} \cup \{\O\}$, where $\O$ is the point at infinity, is an abelian group under the operation $\oplus$ with identity $\O$. 
\end{prop}
\textbf{Proof}\\
Binary operation $\oplus$ which preserves $E(\R)$. Check axioms.

1) Identity: $P \oplus \O = \O\oplus P = P$ for all $P$.

2) Inverses: $P \oplus -P = \O$

3) Abelian: Computing secant lines with different order of endpoints gives the same line, so $\oplus$ commutes

4) Associativity: In principle, this can be done by algebra with exhaustive case study. Alternatively, \\
$\to$ 4.1) do this in projective geometry, use Pascal's theorem\\
$\to$ 4.2) Develop theory of algebraic curves enough, it becomes obvious (tensor product with Picard group, that is a group and is associative, so this is associative)

\subsection{Introducing other fields}
\begin{rmk}
    We don't actually care about $E(\R)$, but variations are useful in cryptography
\end{rmk}

\begin{defn}
    $E(\Q) = \{(x, y) \in \R^2 | y^2 = x^3 + ax + b\} \cup \{\O\} \subset E(\R)$
\end{defn}

\begin{rmk}
    It is possible for there to be no rational points and $E(\Q)$ is just $\O$
\end{rmk}

\textbf{Claim}: $E(\Q)$ is a subgroup of $E(\R)$ under $\oplus$

1) $\O \in E(\Q)$ (either by definition of $E(\Q)$ or since $\O$ is (0, 0, 1) in projective geometry

2) $P \in E(\Q) \implies -P \in E(\Q)$, obvious since $-P = (x_1, -y_1)$

3) $P, Q \in E(\Q) \implies P\oplus Q \in E(\Q)$. All special cases are obvious. For the general case, all of the suboperations are closed under rational numbers, so the entire operation is a rational operation.

\begin{rmk}
    A field is a set $K$ with operations $+, \cdot$ satisfying a collection of axioms that satisfy all the expected axioms as under real numbers ($+, -, \cdot, /$)
\end{rmk}

\begin{exm*}
    $R, \Q, \C, \F_p = \Z/p$ where $p$ prime.
\end{exm*}

\begin{rmk}
    Modulus has to be prime since $\Z/n$ can have elements without an inverse (not even integral domain)
\end{rmk}

\begin{defn}
    For field $K$, an elliptic curve over $K$ is $Y^2 = X^3 + aX + b$ where $a, b \in K$ s.t $\Delta_E = 4a^3 + 27b^2 \neq 0$.
\end{defn}

$E(K) = \{(x, y) \in K \times K | Y^2 = X^3 + aX^2 + b \in K\} \cup \{\O\}$ is an abelian group under $\oplus$. 

\begin{exm*}
    $E(\F_p) = \{(x, y) \in \F_p^2 | Y^2 = X^3 + aX^2 + b\pmod{p}\} \cup \{\O\}$\\
    $E = y^2 = x^3 + x + 1, K = \F_p$
\end{exm*}

\begin{tabular}{c|c|c}
    $x$ & $x^3 + x + 1$ & $y$ s.t $y^2 = x^3 + x + 1$\\
    \hline
    0 & 1 & $\pm$ 1\\
    1 & 3 & $X$\\
    2 & 4 & $\pm$2\\
    3 & 3 & $X$\\
    4 & 6 & $X$\\
    5 & 5 & $X$\\
    6 & 6 & $X$\\
\end{tabular}

$E(\F_7) = \{\O, (0, 1), (0, -1), (2, 2), (2, -2)\}$

$(0, 1) \oplus (2, 2)$

$\lambda = \dfrac{2-1}{2-0} = \dfrac{1}{2} = 4$\\
$\implies x_3 = \lambda^2 - x_1 - x_2 = 16 - 0 - 2 = 14 = 0$\\
$\implies y_3 = 1 + 4(0 - 0) = 1$\\
$\implies (0, 1) \oplus (2, 2) = -(0, 1) = (0, -1)$

\subsection{Classifying E} What kind of groups are we getting?

\begin{exm*}
    $E(\F_p)$ is a finite abelian group. $|E(F_p)| \leq p^2 + 1$, but we can do far better, since for each $x$ coordinate can give us at most 2 $y$ coordinates, so $|E(F_p)| \leq 2p + 1$.
\end{exm*}
This bound still isn't best, but it's better

\begin{exm*}
    $E(\R)$ is either $S^1$ or $S^1 \times \Z/2$, where $S^1$ is the circle group under addition of angles.
\end{exm*}
Which one it is is detectable based on how many roots $E$ has. Only 1 compact lie group of dimension 1, which is $S^1$.

\begin{exm*}
    $E(\C)$ is the torus, $S^1 \times S^1$
\end{exm*}

\begin{thm}[Mordell-Weil Theorem]
    $E(\Q)$ is a finitely generated abelian group $\implies E(\Q) \cong \Z^r \times T$, where $r \geq 0$, and $T$ is the torsion group. (which is finite) %Todo: research this
\end{thm}


\begin{exm*}
    $E(\Q) \cong \Z$, there is a point $P_0 \in E(\Q)$ s.t every point in $E(\Q)$ is $nP_o$ for some $n\in \Z$
\end{exm*}
$nP_o$ := $P_o \oplus P_o \oplus \cdots \oplus P_o$ for $n > 0$, or $-P_o \oplus -P_o \oplus \ldots \oplus -P_o$ for $n < 0$.

\begin{thm}[Mazar, 1977]
    $\begin{cases}
        T \cong \Z/n & n = 1, 2, \ldots, 10, 12\\
        T \cong \Z/2 \times \Z/n & n = 2, 4, 6, 8\\
    \end{cases}$
\end{thm}

"Mazar is the best number theorist of the 20th century, but I'm a bit biased" - man advised by Mazar.

What about $r$? Called the rank. $r$ is 0, 50\% of the time, and $r = 1$ 50\%. $r \geq 2$ occurs but rarely. Record $r$ is probably around 30, hypothesis is that $r$ is unbounded.

There are certain algorithms to compute $r$ and $E(\Q)$

\begin{rmk}
    There is a conjectural analytic formula for $r$. Birch and Swinnerton-Dyer %Todo: research https://scholar.rose-hulman.edu/cgi/viewcontent.cgi?article=1028&context=rhumj
\end{rmk}

% \begin{rmk}
%     Silverman + Tate: Rational Points on Elliptic Curves. Should be at the level we can understand, at the cost of using less effective algebraic arguments at times
% \end{rmk}

\section{Elliptic Curves over Finite Fields}
$E : y^2 = x^3 + ax^2 + b$, where $a, b \in \F_p$

How big can $E(\F_p)$ be? 

How to compute?

First approach: for each $x = x_0$, look at $x_0^3 + ax_0 + b$.$= \left(\dfrac{x_0^3 + ax_0^2 + b}{p}\right) + 1$ (Legandre symbol)\\$\implies$ if this is a nonzero square, 2 points. For nonsquare, 0 points. zero, 1 point. 

$|E(\F_p)| = \sum\limits_{x_0=0}^{p-1}\left(\dfrac{x_0^3 + ax_0^2 + b}{p}\right) + 1 + 1 = p + 1 + \sum\limits_{x_0=0}^{p-1}\left(\dfrac{x_0^3 + ax_0^2 + b}{p}\right)+ 1$ 

Since $\left(\dfrac{a}{p}\right)$ is 1 or -1 equally often, expect sum to be fairly small.
\begin{thm}[Riemann Hypothesis for elliptic curves over finite fields]
    $|\sum\limits_{x_0=0}^{p-1}\left(\dfrac{x_0^3 + ax_0^2 + b}{p}\right)| \leq 2\sqrt p$, 
\end{thm}
Really called the Hasse Theorem, but Hasse applied to the Nazi party, and Weston doesn't cite Nazis

$N_p = \#E(\F_p)$\\
$a_p = p + 1 - \#E(\F_p)$\\
$|a_p| \leq 2\sqrt{p}$\\
$|\#E(\F_p) - p - 1| \leq 2\sqrt p$

\begin{rmk}
    \# $E(\F_p) = p + 1$, where everything cancels out, is the supersingular case.
    \# $E(\F_p) = p \to$ "anomolous primes", discrete log problem is really easy to solve %Todo: why
\end{rmk}

\subsection{Algorithms to compute \#E(F)}
Given $E/\F_{101}$, suppose we have $P \in E/\F_{101}$ of order 47. This directly implies that \#$E(\F_p) = 94$.

Why? Riemann hypothesis tells us that the number of points must be within $|\#E(\F_p) - 102| \leq 20 \implies 82 \# \leq \#E(\F_p) \leq 122$. Lagrange's theorem tells us that, since $E/\F_{101}$ is finite, then order of $P$ must divide $\#E(\F_p)$. The only number that satisfies both of these properties is 94.

To compute $\#E(\F_p)$: find orders of elements until Lagrange forces a unique possible field order via Riemann Hypothesis.

How to find orders?\\
1) Shanks Baby Step -- Giant Step (Collision): take big powers and find collision. Going to take $O(\sqrt{p})$, might need to make multiple tries before you get a useful collision

2) Schoof (Elkies + Atkin). Using division polynomials, runs in $O(\log^6 p)$. The constants were originally huge, so you need lots of digits for it to be useful/practical.

\begin{rmk}
    Any finite abelian group can be expressed as the product of finite cyclic groups. $E(\F_p)$ can be a product of at most two cyclic groups: $E(\F_p) \cong \Z/st \times \Z/s$, where $t$ is large and $s$ is small. For example, prime $l | s \approx \dfrac{1}{l^4}$
\end{rmk}

\begin{exm*}
    Another way to look at RH. Take $y^2 = x^3 - 7x - 6$. Vary $p$, count $\#E(F_p)$ for each $p$, and compare to Riemann hypothesis
\end{exm*}

\begin{tabular}{c|c|c}
    $p$ & $\#E(\F_p)$ & $p + 1 - \#E(\F_p) \leq 2\sqrt{p}$\\
    \hline
    2 & - & -\\
    3 & 4 & 0\\
    5 & - & -\\
    7 & 12 & -4\\
    11 & 8 & 4\\
    13 & 16 & -2\\
    17 & 16 & 2\\
    19 & 16 & 4\\
\end{tabular}

Middle columns are all multiples of four, the third column will therefore all be even.

\begin{rmk}
    Wiles (in proving Fermat's Last Theorem) the $a_p$ are the Fourier coefficients of a modular form %Todo: what
\end{rmk}
\begin{rmk}
    $E(\Q)$ infinite $\iff \prod\limits_p \dfrac{p}{\#E(\F_p)} = 0$
\end{rmk}

\subsection{Elliptic Curve Discrete Log Problem (ECDLP)}
\begin{defn}
    Take $P, Q \in E(\F_p)$. Find $n$ such that $Q = n\cdot P$, where $n$ is an additive power using the addition law of $E/F_p$
\end{defn}

\begin{exm*}
    $E/\F_{101}$, $y^2 = x^3 + x + 3$. $P = (46, 83)$, $Q = (31, 63)$
\end{exm*}
How do we find $n$ such that $Q = nP$? $n = 37$ works. In other words, $\log_p Q = 37$

We need a basic algorithm to compute $n \cdot P$ quickly for $P \in E(\F_p), n > 0$. "double and add"

\begin{exm*}
    $E: y^2 = x^3 + 31x + 1000$ over $\F_{32003}$ %"My wife's favorite finite field"
\end{exm*}

Find $P$ on $E(\F_p)$. Try $x = 1 \implies y^2 = 1032$. Compute $\left(\dfrac{1032}{32003}\right) = +1 \implies y$ exists. $y = \pm 21953$. Take $P = (1, 21953)$.

Compute $1297 \cdot P$. Decompose it as a power of 2: 1297 = 1024 + 256 + 16 + 1.

$P = (1, 21953)$. $P + P = (10821, 20322), 4P = 2P + 2P = (\ldots)$

$16P = 8P + 8P = (8878, 16557)$\\
$256P = (19325, 10689)$\\
$1024P = (13434, 22968)$\\
$1297P = 1024P + 256P + 16P + P = (544, 26812)$

\begin{rmk}
    Similar to fast powering, this algorithm can also be adapted to minimize storage requirements.
\end{rmk}

\begin{rmk}
    There is no Fermat's Little Theorem here, because we don't know the order of the group
\end{rmk}

ECDLP: Recover 1297 from (544, 26812) and (1, 21953).

Best known algorithms are collision algorithms taking $O(\sqrt{p})$ steps. These are slow, which are good for cryptographic reasons.

\begin{rmk}
    For regular discrete log problem, there exist subexponential algorithms for general prime $p$. Additionally, there exist this idea of bad primes $p$. Here, the best algorithm is obviously exponential.
\end{rmk}

\begin{rmk}
    In essence, Shor's algorithm is really good at computing orders of elements mod $p$ very quickly
\end{rmk}

\subsection{Collision Algorithms} These are essentially an adaptation of Baby Step -- Giant Step

$S$ finite set, $\#S = N$. Define $f : S \to S$ that is "sufficiently random".

\begin{exm*}
    $S  = \Z/n, f(x) = x^2 + 1$.
\end{exm*}

We are more interested in $S = E(\F_p)$

Given $P, Q \in E(\F_p)$

$F(A) = \begin{cases}
    A + P & x \equiv 1\pmod{3}\\
    2A & x \equiv 2\pmod{3}\\
    A + Q & x \equiv 0\pmod{3}\\
\end{cases}$ \hspace{1em} for $A \in E(\F_p) = (x, y), 0 \leq x \leq p-1$
%If you've taken differential equations, first of all my condolences

\textbf{Idea}: Fix $x_0 \in S$. $x_1 = f(x_0)$, $x_2 = f(x_1), \cdots$

Mapping points to points, and eventually you will have a cycle because we are dealing with a finite set. Call the first point in the cycle you see $x_T$, the last point in the cycle $x_{T + M - 1}$, and then $x_T$ repeats as $x_{T + M}$, where $T$ and $M$ are minimum

\begin{rmk}
    In Chapter 5, How large to you expect T to be? $O(\sqrt{N})$
\end{rmk}

\subsection{Pollard's factorization algorithm} Assume we have $n = pq$, $S = \Z /n$, $f(x) = x^2 + 1$. $x_0 = 1$

Suppose $x_{T_n} = x_{T_n + M_n}$ is the first repeat mod $n$, $T_n = O(\sqrt{n})$. Probably, we get a repeat mod $p$ (or $q$) much much sooner: $x_{T_p} = x_{T_p + M_p}, T_p = O(\sqrt{p}) = O(n^{1/4})$. Take $\gcd(x_{T_p} - x_{T_p + M_p}, n) = p$, and we can probably recover something.

\textbf{Implementation Problems}: You need to compute $\gcd(x_i - x_j, n)$ for every pair $i, j$, because we have no idea where this repeat is going to be. This becomes a huge number as $i$ increases. Additionally, you have to store every point, which is infeasible.

\begin{defn}[Pollard $\rho$--method]
    Traverse twice. Start with $x_0 = y_0$, and compute $x_i = f(x_i - 1)$, $y_i = f(f(y_{i + 1}))$. At each step, compute $\gcd(x_i - y_i, n)$. If it fails, throw it away. If it works, we have $x_T = x_{M + T}$.
\end{defn}

\begin{exm*}
    $n = 31861$, $f = x^2 + 1, x_0 = 1$
\end{exm*}

\begin{tabular}{c|c|c|c}
    $i$ & $x_i$ & $y_i$ & $\gcd(x_i - y_i, n)$\\
    \hline
    0 & 1 & 1 & $n$\\
    1 & 2 & 5 & 1\\
    2 & 5 & 677 & 1\\
    3 & 26 & 29508& 1\\
    4 & 677 &  27909 & 151\\
\end{tabular}

Unless we get unlucky, and $q$ hits at the exact same moment, we have that 151 is a factor of $n$.

Running time depends on the smallest prime factor $O(\sqrt{p}) \stackrel{?}{=} O(n^{1/4})$. If $p$ is much smaller, then it runs much better
\end{document}


