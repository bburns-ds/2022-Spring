% ENORMOUS credit for the foundational formatting
%   of these notes goes to Patrick Lei. 
% Check out his fantastic course notes
%   here: https://git.snopyta.org/abstract/course-notes
% and his website 
%   here: https://www.math.columbia.edu/~plei/index.html

\documentclass[twoside, 10pt]{article}
\usepackage{bbnotes}
\geometry{margin=2cm}

\definecolor{darkblue}{RGB}{0,0,128}
\definecolor{darkred}{RGB}{128,0,0}
\definecolor{darkyellow}{RGB}{96,96,0}
\definecolor{darkgreen}{RGB}{0,128,0}
\definecolor{darkdarkred}{RGB}{64,0,0}

\lstset
{
basicstyle=\ttfamily\scriptsize, 
breaklines=true,
postbreak=\mbox{\textcolor{darkdarkred}{$\hookrightarrow$}\space},
showstringspaces=false
keywordstyle = [1]\bfseries\color{darkred},
keywordstyle = [2]\itshape\color{darkgreen},
keywordstyle = [3]\sffamily\color{darkblue},
keywordstyle = [4]\color{darkyellow},
}

\fancypagestyle{firstpage}
{
   \fancyhf{}
   \fancyfoot[R]{\itshape Page \thepage\ of \pageref{LastPage}}
   \renewcommand{\headrulewidth}{0pt}
}


\fancypagestyle{pages}
{
\fancyhead[C]{\scshape Math 571}
\fancyhead[L]{\scshape Ben Burns}
\fancyhead[R]{\scshape Spring 2022}
\renewcommand{\headrulewidth}{0.1pt}
}

\pagestyle{pages}

\title{Math 571}
\author{Taught by Tom Weston \\ Scribed by Ben Burns}
\affil{UMass Amherst}
\date{Spring 2022}

\newcommand{\F}{\mathbb{F}}
\usepackage{algorithmicx}

\begin{document}

\maketitle\thispagestyle{firstpage}

\tableofcontents

\section{First Exam}
Covers through today
\begin{itemize}
    \item Discrete Logs
    \item Diffie-Hellman Key Exchange
    \item Shanks Algorithm
    \item RSA
    \item Factorization (p-1 method, trial division, difference of squares)
\end{itemize}
\begin{exm}
    Fiven a few explicity congruences $c_i \equiv a_i^2\pmod{n}$ explain how you can find a factor of $n$
\end{exm}

Practice exam by Thursday

Format of Exam: Around 4 questions, 2 proof, 2 computation, no calculator

\section{Factorization by difference of squares}
(1) Find lots of congruences $a_1^2 \equiv c_i \pmod{n}$ with $c_i$ product of small primes. Fix small number $B$, and require all prime factors $p \leq B$\\
(2) Elimination: Find a subset of these congruences which multiply to give $x^2 \equiv y^2 \pmod{n}$ ($\F_2$ linear algebra)\\
(3) Compute $\gcd(x\pm y, n)$, hope its a proper factors of $n$

Review finding kernel, Gaussian Elimination $\F_2$, book of (2)

\textbf{Algorithm}: Find numbers $a$ such that $a^2 \pmod{n$} is a product of small primes

$m = \left\lfloor \sqrt{n}\right\rfloor + 1$. Try $a = m, m + 1, m+2, \cdots$, $a^2 \pmod{n} = a^2 - n$. 

This is relatively small since $a \approx\sqrt n$, so has a better chance of factoring into small primes

$Q(x) = x^2 - n$

Looking at $x = m, m+1, m+2$, we find $x^2 \equiv Q(x) \pmod{n}$, where $x^2 = a_i$ and $Q(x) = c_i$.

\textbf{Problem}: Given $n, a$, how do you determine if $Q(a) = a^2 - n$ is a product of small primes without factoring?

%Todo: What is a KMatrixSpace

\begin{rmk}
    Roughly half of primes will never be factors of $Q(a)$
\end{rmk}
Why? Suppose $p | Q(a)$ for some $a$. Then $p | a^2 - n \implies a^2 \equiv n \pmod{p} \implies$ $n$ is a square mod $p$.

Fix odd prime $p$

\begin{defn}
    Given $t\in \F_p$ such that $t\neq 0$, we say $t$ is a quadratic residue mod $p$ if $\exists s \in \F_p$ such that $s\equiv s^2 \pmod{p}$
\end{defn}
For example, $p=11$

"Squares are always squares" - :D

There are always exactly $\dfrac{p-1}{2}$ squares and $\dfrac{p-1}{2}$ non-squares 
%Todo: prove the first half of the squares will be distinct

\begin{defn}[Legendre Symbol]
    $\left(\dfrac{t}{p}\right) = +1$ if $t$ square mod $p$, $-1$ if $t$ non-square, $0$ if $t = 0 \pmod{p}$.
\end{defn}

\begin{rmk}
    The quadratic resides of $\F_p$ are the even powers of any generator $g$.
\end{rmk}
Fix a generator $g\in \F_p$. Fix $t\in \F_p$. Write $t = g^e$ for some $e$ s.t $0\leq e \leq p-1$. If $e$ is even then $t = (g^{e/2})^2 \implies \left(\dfrac{t}{p}\right) = 1$. Since this already gives $(p-1)/2$ squares for $e = 0, 2, 4, \ldots p-3$, so it follows that $e$ odd $\implies \left(\dfrac{t}{p}\right) = -1$.

\begin{defn}[Properties of Legendre Symbol]
    (1) $\left(\dfrac{t}{p}\right) \equiv t^{\dfrac{p-1}{2}} \pmod{p}$\\
    (2) $\left(\dfrac{st}{p}\right) = \left(\dfrac{s}{p}\right)\left(\dfrac{t}{p}\right)$\\
    (3) $\left(\dfrac{-1}{p}\right) = (-1)^{\dfrac{p-1}{2}} = 1$ if $p\equiv 1\pmod{4}$, $-1$ is $p\equiv 3\pmod{4}$
\end{defn}
Proof:\\ 
%Todo: fill out
(1) FLT for squares, polynomial counting argument for non-squares\\
(2) right side of (1) is multplicative, so left side has to be as well\\
(3) 

\begin{defn}[Quadratic Reciprocity Law]
    Let $p, q$ be distinct odd positive primes. Then $\left(\dfrac{p}{q}\right) = \dfrac{q}{p}(-1)^{\dfrac{p-1}{2}\dfrac{q-1}{2}}$.
\end{defn}
Equivalently, if either $p$ or $q$ is congruent to 1 mod 4, then the reciprocity holds. Else, they are negations.
\begin{defn}[Quadratic Sieve] 
    brr
\end{defn}
\end{document}