\documentclass[twoside, 10pt]{article}
\usepackage{bbnotes}
\geometry{margin=2cm}

\fancypagestyle{pages}
{
\fancyhead[C]{\scshape Math 523H}
\fancyhead[L]{\scshape Burns, Wei}
\fancyhead[R]{\scshape Spring 2022}
\renewcommand{\headrulewidth}{0.1pt}
}

\newcommand{\dom}{\text{dom }}

\pagestyle{pages}

\begin{document}
\section*{Chapter 4: Completeness Axiom}
\begin{defn}[Max and Min]
Let $S$ be a nonemtpy subset of $\R$.
\begin{enumerate}
    \item If $S$ contains a largest element $s_0$ (that is, $s_0$ belongs to $S$ and $s\le s_0$ for all $s\in S$, then we call $s_0$ the \emph{maximum} of $S$ and write $s_0=\max S$.
    \item If $S$ contains a smallest element, then we call the smallest element the \emph{minimum} of $S$ and write it as $\min S$
\end{enumerate}
\end{defn}

\begin{defn}[Bounds]
Let $S$ be a nonempty subset of $\R$.
\begin{enumerate}
    \item If a real number $M$ satisfies $s\le M$ for all $s\in S$, then $M$ is called an \emph{upper bound of $S$} and the set $S$ is said to be \emph{bounded above}
    \item If a real number $m$ satisfies $m\le s$ for all $s\in S$, then $m$ is called an \emph{lower bound of $S$} and the set $S$ is said to be \emph{bounded below}
    \item The set $S$ is said to be \emph{bounded} if it is bounded above and below. Thus $S$ is bounded if there exist real numbers $m$ and $M$ such that $S\subseteq [m, M]$.
\end{enumerate}
\end{defn}

\begin{defn}[sup and inf]
Let $S$ be a nonempty subset of $\R$.
\begin{enumerate}
    \item If $S$ is bounded above and $S$ has a least upper bound, then we will call it the \emph{supremum of} $S$ and denote it by $\sup S$.
    \item if $S$ is bounded below and $S$ has a greatest lower bound, then we will call it the \emph{infimum of} $S$ and denote it by $\inf S$.
\end{enumerate}
\end{defn}

\begin{thm} [Completeness Axiom]
Every nonempty subset $S$ of $\R$ that is bounded above has a least upper bound. In other words, $\sup S$ exists and is a real number.
\end{thm}

\begin{cor}
Every nonempty subset $S$ of $\R$ that is bounded below has a greatest lower bound $\inf S$.
\end{cor}

\begin{thm} [Archimedean Property]
If $a > 0$ and $b > 0$, then for some positive integer $n$, we have $na > b$
\end{thm}

\begin{thm} [Density of $\Q$]
If $a,b \in \R$ and $a < b$, then there is a rational $r\in\Q$ such that $a < r < b$.
\end{thm}

\section*{Chapter 7: Limits of Sequences}
\begin{defn}[Convergence]
A sequence $(s_n)$ of real numbers is said to converge to the real number $s$ provided that for each $\epsilon > 0$ there exists a number $N$ such that $n > N$ implies $|s_n - s| \leq \epsilon$. A sequence that does not converge to some real number is said to diverge.
\end{defn}

\section*{Chapter 8: Proofs}
All that you need to know here is the idea of computationally deriving your epsilon to use in your proof.

\section*{Chapter 9: Limit Theorems}
\begin{thm}
Convergent sequences are bounded
\end{thm}

\begin{thm}
If the sequence $(s_n)$ converges to $s$ and $k$ is in $\R$, then the sequence $(ks_n)$ converges to $ks$. That is, $\lim(ks_n) = k\cdot\lim(s_n)$
\end{thm}

\begin{thm}
If $(s_n)$ converges to $s$ and $(t_n)$ converges to $t$, then $\lim(s_n + t_n)$ converges to $s + t$. That is, $\lim(s_n + t_n) = \lim s_n + \lim t_n$
\end{thm}

\begin{lem}
If $(s_n)$ converges to $s$, if $s_n \neq 0$ for all $n$, and if $s \neq 0$, then $(1/s_n)$ converges to $1/s$. 
\end{lem}

\begin{thm}
Suppose $(s_n)$ converges to $s$ and $(t_n)$ converges to $t$. If $s \neq 0$ and $s_n \neq 0$ for all $n$, then $(t_n/s_n$ converges to $t/s$/
\end{thm}

\begin{thm} Basic Examples\\
(a) $\lim_{n \to \infty} (\dfrac{1}{n^p}) = 0$ for $p > 0$\\
(b) $\lim_{n \to \infty} a^n = 0$ if $|a| < 1$.\\
(c) $\lim(n^{1/n}) = 1$.\\
(d) $\lim_{n \to \infty}(a^{1/n}) = 1$ for $a > 0$
\end{thm}

\begin{defn}
For a sequence $(s_n)$, we write $\lim s_n = + \infty$ provided for each $M > 0$ there is a number N such that $n > N$ implies $s_n > M$.
\end{defn}

\begin{thm}
Let $(s_n)$ and $(t_n)$ be sequences such that $\lim s_n = +\infty$ and $\lim t_n > 0$ [$\lim t_n$ can be finite or $+\infty$]. Then $\lim s_nt_n = +\infty$.
\end{thm}

\begin{thm}
For a sequence $(s_n)$ of positive real numbers, we have $\lim s_n = +\infty$ if and only if $\lim(\dfrac{1}{s_n} = 0$
\end{thm}

\section*{Chapter 10: Monotonic \& Cauchy Sequences}
\begin{defn}
A sequence $(s_n)$ of real numbers is called an \textbf{increasing sequence} if $s_n \leq s_{n+1}$ for all $n$, and is called a \textbf{decreasing sequence} if $s_n \geq s_{n+1}$ for all $n$
\end{defn}

\begin{thm}
All bounded monotone sequences converge
\end{thm}

\begin{thm}
Unbounded increasing sequences have a limit of $s_n = +\infty$, and unbounded decreasing has $\lim s_n = -\infty$
\end{thm}

\begin{cor}
If $(s_n)$ is a monotone sequence, then the sequence either converges, diverges to $+\infty$, or diverges to $-\infty$. Thus $\lim s_n$ is always meaningful for monotone sequences.
\end{cor}

\begin{defn}
Let $(s_n)$ be any sequence of real numbers. We define \begin{equation*}
    \limsup{s_n} = \lim\limits{N \to \infty} \sup \{s_n: n > N\} = \sup S
\end{equation*}
and 
\begin{equation*}
    \liminf{s_n} = \lim\limits{N \to \infty} \inf\{s_n : n > N\} = \inf S
\end{equation*}
\end{defn}

\begin{thm}
Let $(s_n)$ be a sequence in $\R$.\\
(i) If $\lim s_n$ is defined [as a real number, $+\infty$, or $-\infty$], then $\liminf s_n = \lim s_n = \limsup s_n$\\
(ii) If $\lim inf s_n = \limsup s_n$, then $\lim s_n$ is defined and $\lim s_n = \lim inf s_n = \limsup s_n$
\end{thm}

\begin{defn}
A sequence $(s_n)$ or real numbers is called a Cauchy sequence if for each $\epsilon > 0$ there exists a number $N$ such that $m, n > N$ implies $|s_n - s_m| < \epsilon$
\end{defn}

\begin{lem}
Convergent sequences are Cauchy sequences.
\end{lem}

\begin{lem}
Cauchy sequences are bounded
\end{lem}

\begin{thm}
A sequence is a convergent sequence if and only if it is a Cauchy sequence.
\end{thm}

\section*{Chapter 11: Subsequences}
\begin{defn}
A sequence $(s_n)$ of real numbers is called an \emph{increasing sequence} if $s_n\le s_{n+1}$ for all $n$, and $(s_n)$ is called \emph{decreasing sequence} if $s_n\ge s_{n+1}$ for all $n$. Note that if $(s_n)$ is increasing, then $s_n\le s_m$ whenever $n< m$. A sequence that is increasing or decreasing will be called a \emph{monotone sequence} or a \emph{monotonic sequence}.
\end{defn}

\begin{thm}
All bounded monotone sequence converge.
\end{thm}

\begin{thm}
(i) If $(s_n)$ is an unbounded increasing sequence, then $\lim s_n = +\infty$\\
(ii) If $(s_n)$ is an unbounded decreasing sequence, then $\lim s_n = -\infty$
\end{thm}

\begin{cor}
If $(s_n)$ is a monotone sequence, then the sequence either converges, diverges to $+\infty$, or diverges to $-\infty$. Thus $\lim s_n$ is always meaningful for monotone sequences.
\end{cor}

\begin{defn}
Let $(s_n)$ be a sequence in $\R$. We define
$$ \limsup s_n = \lim_{N\to\infty} \sup\{s_n : n > N\} $$
and
$$ \liminf s_n = \lim_{N\to\infty} \inf\{s_n : n > N\} $$
\end{defn}

\begin{thm}
Let $(s_n)$ be a sequence in $\R$.\\
(i) If $\lim s_n$ is defined [as a real number, $+\infty$ or $-\infty$], then $\liminf s_n = \lim s_n = \limsup s_n$.\\
(ii) If $\liminf s_n = \limsup s_n$, then $\lim s_n$ is defined and $\lim s_n = \liminf s_n = \limsup s_n$.
\end{thm}

\begin{defn}
A sequence $(s_n)$ of real numbers i called a \emph{Cauchy sequence} if\\
for each $\epsilon > 0$ there exists a number $N$ such that $m,n>N$ implies $|s_n - s_m| < \epsilon$.
\end{defn}

\begin{lem}
Convergent sequences are Cauchy sequences.
\end{lem}

\begin{lem}
Cauchy sequences are bounded.
\end{lem}

\begin{thm}
A sequence is a convergent sequence if and only if it is a Cauchy sequence.
\end{thm}

\section*{Chapter 12: $\limsup$ and $\liminf$}
\begin{defn}
Let $(s_n)$ be any sequence of real numbers, and let $S$ be the set of subsequential limits of $(s_n)$. Recall \begin{equation*}
    \limsup{s_n} = \lim\limits{N \to \infty} \sup \{s_n: n > N\} = \sup S
\end{equation*}
and 
\begin{equation*}
    \liminf{s_n} = \lim\limits{N \to \infty} \inf\{s_n : n > N\} = \inf S
\end{equation*}
\end{defn}

\begin{thm}
If $(s_n)$ converges to a positive real number $s$ and $(t_n)$ is any sequence, then \begin{equation*}
    \limsup s_nt_n = s\cdot \limsup t_n
\end{equation*}
Here we allow the conventions $s \cdot (+\infty) = +\infty$ and $s\cdot(-\infty) = -\infty$ for $s > 0$.
\end{thm}
Proof sketch: handle the three cases separately using subsequences, doing the inequality left to right, then right to left using $\lim\dfrac{1}{s_n} = \dfrac{1}{s}$

\begin{thm}
Let $(s_n)$ be any sequence of nonzero real numbers. Then we have $\liminf\left|\dfrac{s_{n+1}}{s_n}\right| \leq \liminf |s_n|^{1/n}\leq \limsup |s_n|^{1/n} \leq \limsup\left|\dfrac{s_{n+1}}{s_n}\right|$
\end{thm}
Middle is obvious, first and third have similar proofs

\begin{cor}
If $\lim\left|\dfrac{s_{n+1}}{s_n}\right|$ exists [and equals L], then $\lim\left|s_n\right|^{1/n}$ exists [and equals $L$].
\end{cor}

\section*{Chapter 14: Series}
\begin{defn}
We say a series $\sum a_n$ satisfies the \textbf{Cauchy criterion} if its sequence $(s_n)$ of partial sums is a Cauchy sequence.
\end{defn}

\begin{thm}
A series converges if and only if it satisfies the Cauchy criterion.
\end{thm}

\begin{cor}
If a series $\sum a_n$ converges, then $\lim a_n = 0$
\end{cor}

\begin{defn}[Comparison Test]
Let $\sum a_n$ be a series where $a_n geq 0$ for all $n$.\\
(i) If $\sum a_n$ converges and $|b_n| \leq a_n$ for all $n$, then $\sum b_n$ converges.\\
(ii) If $\sum a_n$ = $+ \infty$ and $b_n \geq a_n$ for all $n$, then $\sum b_n = +\infty$.
\end{defn}

\begin{cor}
Absolutely convergent series are absolutely convergent
\end{cor}

\begin{defn}[Ratio Test]
A series $\sum a_n$ of nonzero terms\\
(i) converges absolutely if $\limsup\left|\dfrac{a_{n+1}}{a_n}\right| < 1$\\
(ii) diverges if $\liminf\left|\dfrac{a_{n+1}}{a_n}\right|$\\
(iii) Otherwise $\liminf\left|\dfrac{a_{n+1}}{a_n}\right| \leq 1 \leq \limsup\left|\dfrac{a_{n+1}}{a_n}\right|$ and the test gives no information.
\end{defn}

\begin{defn}[Root Test]
Let $\sum a_n$ be a series and let $\alpha = \limsup |a_n|^{1/n}$. The series $\sum a_n$\\
(i) converges absolutely if $\alpha < 1$,\\
(ii) diverges if $\alpha >1$.\\
(iii) Otherwise $\alpha = 1$ and the test gives no information.
\end{defn}

\section*{Chapter 15: Alternating Series and Integral Test}
\begin{thm}
$\sum \dfrac{1}{n^p}$ converges if and only if $p > 1$
\end{thm}

\begin{thm}[Alternating Series]
If $a_1 \geq a_2 \geq \ldots a_n \leq \ldots \leq 0$ and $\lim a_n = 0$, then the alternating series $\sum (-1)^n+1a_n$ converges. Moreoever, the partial sums $s_n$ from $k = 1$ to $n$ satisfy $|s-s_n| \leq a_n$ for all $n$.

\end{thm}
\section*{Chapter 17: Continuity}
\begin{defn}
Let $f$ be a real-valued function whose domain is a subset of $\R$. The function $f$ is \emph{continuous at $x_0$} if, for every sequence $(x_n)$ in dom($f$) converging to $x_0$, we have $\lim_nf(x_n)=f(x_0)$. If $f$ is continuous at each point of a set $S\subseteq $ dom($f$), then $f$ is said to be \emph{continuous on $S$}. THen function $f$ is said to be \emph{continuous} if it is continuous on dom($f$).
\end{defn}

\begin{thm}
Let $f$ be a real-valued function whose domain is a subset of $\R$. Then $f$ is continuous at $x_0\in\dom(f)$ if and only if:

for each $\epsilon > 0$ there exists $\delta > 0$ such that $x\in\dom(f)$ and $|x-x_0|<\delta$ imply $|f(x)-f(x_0)| < \epsilon$.
\end{thm}

\begin{thm}
Let $f$ be a real-valued function with $\dom(f) \subseteq \R$. If $f$ is continuous at $x_0$ in $\dom(f)$, then $|f|$ and $kf, k\in\R$, are continuous at $x_0$.
\end{thm}

\begin{thm}
Let $f$ and $g$ be real-valued functions that are continuous at $x_0$ in $\R$.\\
Then\\
    (i) $f+g$ is continuous at $x_0$\\
    (ii) $fg$ is continouos at $x_0$\\
    (iii) $f/g$ is continuous at $x_0$ if $g(x_0)\neq 0$
\end{thm}

\begin{thm}
If $f$ is continuous at $x_0$ and $g$ is continuous at $f(x_0)$, then the composite function $g\circ f$ is continuous at $x_0$.
\end{thm}

\section*{Chapter 20: Limits of Functions}
\begin{defn}
Let $S$ be a subset of $\R$, let $a$ be a real number or symbol $\infty$ or $-\infty$ that is the limit of some sequence in $S$, and let $L$ be a real number or symbol $+\infty$ or $-\infty$. We write $\lim_{x\to a^s}f(x)=L$ if 
\begin{equation*}
    \text{$f$ is a function defined on $S$,}
\end{equation*}
and
\begin{equation*}
    \text{for every sequence $(x_n)$ in $S$ with limit $a$, we have $\lim_{n\to\infty}f(x_n) = L$.}
\end{equation*}
\end{defn}

\begin{rmk}\ \\
(a) From the first definition of the continuity chapter, we see that a function $f$ is continuous at $a$ in $\dom(f) = S$ if and only if $\lim_{x\to a^S} = f(a)$.\\
(b) Observe that limits, when they exist, are unique. This follows from the second statement in the previous definition, since limits of sequences are unique.
\end{rmk}

\begin{defn}.
\begin{enumerate}
    \item For $a\in\R$ and a function $f$ we write $\lim_{x\to a}f(x)=L$ provided $\lim_{x\to a^S}f(x)=L$ for some set $S= J\setminus \{a\}$ where $J$ is an open interval containing $a$. $\lim_{x\to a}f(x)$ is called the [\emph{two-sided}] \emph{limit of $f$ at $a$}. Note $f$ need not be defined at $a$ and, even if $f$ is defined at $a$, the value $f(a)$ need not equal $\lim_{x\to a}f(x)$. In fact, $f(a) = \lim_{x\to a}f(x)$ iff $f$ is defined on an open interval containing $a$ and $f$ is continuous at $a$.
    
    \item For $a\in\R$ and a function $f$ we write $\lim_{x\to a^+}f(x) = L$ provided $\lim_{x\to a^S}f(x) = L$ for some open interval $S= (a,b)$. $\lim_{x\to a^+}f(x)$ is the \emph{right-hand limit of $f$ at $a$}. Again $f$ need not be defined at $a$.
    
    \item For $a\in\R$ and a function $f$ we write $\lim_{x\to a^-}f(x) = L$ provided $\lim_{x\to a^S}f(x) = L$ for some open interval $S= (c, a)$. $\lim_{x\to a^-}f(x)$ is the \emph{right-hand limit of $f$ at $a$}.
    
    \item For a function $f$ we write $\lim_{x\to\infty}f(x) = L$ provided $\lim_{x\to\infty^S} f(x) = L$ for some interval $S = (c, \infty)$. Likewise, we write $\lim_{x\to-\infty}f(x) = L$ provided $\lim_{x\to-\infty^S}f(x) = L$ for some interval $S=(-\infty, b)$
\end{enumerate}
\end{defn}

\begin{thm}
Let $f_1$ and $f_2$ be functions for which the limits $L_1 = \lim_{x\to a^S}f_1(x)$ and $L_2=\lim_{x\to a^S}f_2(x)$ exist and are finite. Then
\begin{enumerate}
    \item $\lim_{x\to a^S}(f_1+f_2)(x)$ exists and equals $L_1 + L_2$
    \item $\lim_{x\to a^S}(f_1f_2)(x)$ exists and equals $L_1L_2$
    \item $\lim_{x\to a^S}(f_1/f_2)(x)$ exists and equals $L_1/L_2$ provided $L_2\neq 0$ and $f_2(x)\neq 0$ for $x\in S$
\end{enumerate}
\end{thm}

\begin{thm}
Let $f$ be a function for which the limit $L=\lim_{x\to a^S}f(x)$ exists and is finite. If $g$ is a function defined on $\{f(x): x\in S\}\cup\{L\}$ that is continuous at $L$, then $\lim_{x\to a^S}g\circ f(x)$ exists and equals $g(L)$.
\end{thm}

\begin{cor}
Let $f$ be a function defined on $J\setminus \{a\}$ for some open interval $J$ containing $a$, and let $L$ be a real number. Then $\lim_{x\to a}f(x) = L$ if and only if
\begin{equation*}
    \text{for each $\epsilon > 0$ there exists $\delta > 0 $ such that $0<|x-a|<\delta$ implies $|f(x)-L|<\epsilon$}
\end{equation*}
\end{cor}

\begin{cor}
Let $f$ be a function defined on some interval $(a, b)$, and let $L$ be a real number. Then $\lim_{x\to a^+}f(x) = L$ if and only if
\begin{equation*}
    \text{for each $\epsilon > 0$ there exists $\delta >0$ such that $a < x < a + \delta$ implies $|f(x) - L| < \epsilon$}
\end{equation*}
\end{cor}

\begin{thm}
Let $f$ be a function defined on $J \setminus {a}$ for some open interval $J$ containing $a$. Then $\lim_{x \to a}f(x)$ exists if and only if the limits $\lim_{x \to a^+} f(x)$ and $\lim_{x \to a^-}f(x)$ both exist and are equal, in which case all three limits are equal.
\end{thm}
\end{document}